\documentclass[twoside]{article}
\usepackage{../../estilo-ejercicios}
\newcommand{\colapso}{{\searrow\!\!\!\!\searrow}}
%--------------------------------------------------------
\begin{document}

\title{Ejercicios de Teoría Geométrica de Grupos}
\author{Javier Aguilar Martín}
\maketitle

\section{Grupos finitamente generados como espacios métricos}

\begin{ejercicio}{1.1}
Dibujar los grafos de Cayley de:
\begin{enumerate}
\item $D_n=\gene{a,b\mid a^n=b^2=1,aba=b}$.
\item $D_\infty=\gene{a,b\mid b^2=1,aba=b}$.
\item Icosahedro: $\gene{a,b,c\mid a^2=b^2=c^2=(ab)^3=(ca)^5=(bc)^2=1}$.
\end{enumerate}

\end{ejercicio}
\begin{solucion}

\end{solucion}

\newpage

\begin{ejercicio}{1.2}
Supongamos que $N\trianglelefteq G$ es finito y $G/N$ es finitamente generado. Entonces $G$ es quasi-isométrico a $G/N$. 
\end{ejercicio}
\begin{solucion}


\end{solucion}




\newpage

\begin{ejercicio}{1.3}
En $\Z$ consideramos las métricas
\[
d(x,y)=|x-y|,\quad d'(x,y)=|x-y|+\log(|x-y|), d'(x,x)=0
\]
y sea $Id:(\Z,d)\to(\Z,d')$. Demostrar que
\begin{enumerate}[a)]
\item para todo $\lambda>1$ existe $c\geq 0$ tal que $Id$ es una $(\lambda,c)$-quasi-isometría,
\item no existe $c\geq 0$ tal que $Id$ es una $(1,c)$-quasi-isometría.
\end{enumerate}
\end{ejercicio}
\begin{solucion}

\end{solucion}

\newpage

\begin{ejercicio}{1.4}
Sea $T_d$ el árbol $d$-regular. Demostrar que $T_3$ es quasi-isométrico a $T_4$.
\end{ejercicio}
\begin{solucion}
EN EL ENUNCIADO HAY UN DIBUJO. NO SÉ SI ES IMPORTANTE EL NÚMERO DE VÉRTICES, INTENTARLO DE CUALQUIER FORMA. 
\end{solucion}

\newpage

\begin{ejercicio}{1.5}
$\Z$ no es quasi-isométrico a $\Z^2$.
\end{ejercicio}
\begin{solucion}
SE PUEDE HACER FÁCIL CON EL SIGUIENTE TEMA, PERO INTENTAR CON LO DE ESTE
\end{solucion}

\newpage

\section{Invariantes de quasi-isometría: crecimiento}

\begin{ejercicio}{2.1}
Probar que $\beta_{\Z^k}\sim n^k$.
\end{ejercicio}
\begin{solucion}

\end{solucion}

\newpage

\begin{ejercicio}{2.2}
Calcular el crecimiento de $\left\{\begin{pmatrix}
1 & x & z\\
0 & 1 & y\\
0 & 0 & 1
\end{pmatrix}\middle\vert x,y,z\in\Z\right\}$. Ayuda: $a=\begin{pmatrix}
1 & 1 & 0\\
0 & 1 & 0\\
0 & 0 & 1
\end{pmatrix}$ y $b=\begin{pmatrix}
1 & 0 & 0\\
0 & 1 & 1\\
0 & 0 & 1
\end{pmatrix}$ generan el grupo, pero es más fácil si se considera también $c=\begin{pmatrix}
1 & 0 & 1\\
0 & 1 & 0\\
0 & 0 & 1
\end{pmatrix}$ como generador. 
\end{ejercicio}
\begin{solucion}

\end{solucion}

\newpage

\begin{ejercicio}{2.3}
Probar las siguientes afirmaciones:
\begin{enumerate}
\item $\beta_{(G,X)}(n+m)\leq \beta_{(G,X)}(n)\beta_{(G,X)}(m)$.
\item $w(G,X)=\limsup_{n\to \infty}\sqrt[n]{\beta(G,X)(n)}$ es un límite. 
\item $G$ es de crecimiento exponencial si y solo si $w(G,X)>1$.
\end{enumerate}
\end{ejercicio}
\begin{solucion}

\end{solucion}

\newpage

\begin{ejercicio}{2.4}
Un grupo finitamente generado es \textbf{promediable} si existe una sucesión $\{F_n\}_{n\in\N}$ de conjuntos finitos tal que $\lim_{n\to\infty}\frac{|\partial F_n|}{|F_n|}=0$, donde $\partial F_n=\{x\in G\mid x\notin F, d(x,F)=1\}$.

Probar que si el crecimiento de $G$ no es exponencial, entonces $G$ es promediable. 
\end{ejercicio}
\begin{solucion}

\end{solucion}

\newpage

\section{Invariantes de quasi-isometría: finales}

EN PRINCIPIO NO HAY EJERCICIOS, TAL VEZ PONER AQUÍ ALGUNO DE LOS EJEMPLOS QUE SE DEJA COMO EJERCICIO

\section{Grupos finitamente presentados}
\begin{ejercicio}{3.1}

Sea $G=\gene{x,y\mid xy^2=y^3x, yx^2=x^3y}$. Probar que $G$ es trivial.
\end{ejercicio}
\begin{solucion}

\end{solucion}

\newpage

\begin{ejercicio}{3.2}
$\Z^n$ tiene función de Dehn cuadrática. 
\end{ejercicio}
\begin{solucion}
\end{solucion}

\newpage

\begin{ejercicio}{1.12}

\end{ejercicio}
\begin{solucion}

\end{solucion}

\newpage

\begin{ejercicio}{1.13}
Indicar si alguna de las siguientes aplicaciones entre los complejos indicados más abajo es simplicial:
\begin{enumerate}[(a)]
\item
\[
\begin{tikzpicture}[line cap=round,line join=round,>=triangle 45,x=1.0cm,y=1.0cm]
\clip(0.068,-1.) rectangle (12.334666666666664,3);
\draw [line width=1.pt] (3.,0.)-- (6.,0.);
\draw [line width=1.pt] (6.,0.)-- (4.5,2.5980762113533165);
\draw [line width=1.pt] (4.5,2.5980762113533165)-- (3.,0.);
\draw [line width=1.pt] (7.,0.)-- (10.,0.);
\draw [line width=1.pt] (10.,0.)-- (8.5,2.5980762113533165);
\draw [line width=1.pt] (8.5,2.5980762113533165)-- (7.,0.);
\draw [line width=1.pt] (5.25,1.2990381056766587)-- (3.75,1.2990381056766582);
\draw [line width=1.pt] (3.75,1.2990381056766582)-- (4.5,0.);
\draw [line width=1.pt] (4.5,0.)-- (5.25,1.2990381056766587);
\draw (4.3,0.06) node[anchor=north west] {$v_1$};
\draw (2.8,0.06) node[anchor=north west] {$v_0$};
\draw (5.8,0.06) node[anchor=north west] {$v_2$};
\draw (3.1,1.506666666666661) node[anchor=north west] {$v_4$};
\draw (5.3,1.5173333333333276) node[anchor=north west] {$v_3$};
\draw (4.3,2.9573333333333234) node[anchor=north west] {$v_5$};
\draw (6.8,0.06) node[anchor=north west] {$w_0$};
\draw (9.6,0.06) node[anchor=north west] {$w_1$};
\draw (8.2,2.96) node[anchor=north west] {$w_2$};
\draw (1,3.0) node[anchor=north west] {$v_0\to w_0$};
\draw (1.,2.6) node[anchor=north west] {$v_1\to w_1$};
\draw (1.,2.2) node[anchor=north west] {$v_2\to w_1$};
\draw (1.,1.8) node[anchor=north west] {$v_3\to w_1$};
\draw (1.,1.4) node[anchor=north west] {$v_4\to w_2$};
\draw (1.,1.) node[anchor=north west] {$v_5\to w_1$};
\end{tikzpicture}
\]
\item
\[
\begin{tikzpicture}[line cap=round,line join=round,>=triangle 45,x=1.0cm,y=1.0cm]
\clip(0.068,-1.) rectangle (12.334666666666664,3);
\draw [line width=1.pt] (3.,0.)-- (6.,0.);
\draw [line width=1.pt] (6.,0.)-- (4.5,2.5980762113533165);
\draw [line width=1.pt] (4.5,2.5980762113533165)-- (3.,0.);
\draw [line width=1.pt] (7.,0.)-- (10.,0.);
\draw [line width=1.pt] (10.,0.)-- (8.5,2.5980762113533165);
\draw [line width=1.pt] (8.5,2.5980762113533165)-- (7.,0.);
\draw [line width=1.pt] (9.25,1.2990381056766587)-- (7.75,1.2990381056766582);
\draw [line width=1.pt] (7.75,1.2990381056766582)-- (8.5,0.);
\draw [line width=1.pt] (8.5,0.)-- (9.25,1.2990381056766587);
\draw [line width=1.pt] (4.5,0.)-- (4.5,2.6);
\draw (4.3,0.06) node[anchor=north west] {$v_2$};
\draw (2.8,0.06) node[anchor=north west] {$v_0$};
\draw (5.8,0.06) node[anchor=north west] {$v_3$};
\draw (4.3,2.9573333333333234) node[anchor=north west] {$v_1$};
\draw (6.8,0.06) node[anchor=north west] {$w_0$};
\draw (8.3,0.06) node[anchor=north west] {$w_5$};
\draw (9.6,0.06) node[anchor=north west] {$w_4$};
\draw (8.2,2.96) node[anchor=north west] {$w_2$};
\draw (7.1,1.506666666666661) node[anchor=north west] {$w_1$};
\draw (9.3,1.5173333333333276) node[anchor=north west] {$w_3$};
\draw (1,3.0) node[anchor=north west] {$v_0\to w_0$};
\draw (1.,2.6) node[anchor=north west] {$v_1\to w_4$};
\draw (1.,2.2) node[anchor=north west] {$v_2\to w_2$};
\draw (1.,1.8) node[anchor=north west] {$v_3\to w_1$};
\end{tikzpicture}
\]

\end{enumerate}
\end{ejercicio}
\begin{solucion}
\begin{enumerate}[(a)]
\item Claramente sí lo es, porque cualquier conjunto de vértices va a un símplice del complejo imagen. Así que da igual cuál sea el conjunto origen siempre que los vértices se transformen en vértices.
\item No lo es puesto que el conjunto de vértices $\{v_0, v_2\}$, que forman la arista $(v_0v_2)$, se mapea en $\{w_0,w_2\}$, pero la arista $(w_0w_2)$ no forma parte del complejo simplicial.
\end{enumerate}
\end{solucion}

\newpage

\begin{ejercicio}{1.14}
Sea $f : Δ^2 \to Δ^2$ la aplicacion simplicial defina por

\[
\begin{tikzpicture}[line cap=round,line join=round,>=triangle 45,x=1.0cm,y=1.0cm]
\clip(0.068,-1.) rectangle (12.334666666666664,3);
\draw [line width=1.pt] (3.,0.)-- (6.,0.);
\draw [line width=1.pt] (6.,0.)-- (4.5,2.5980762113533165);
\draw [line width=1.pt] (4.5,2.5980762113533165)-- (3.,0.);
\draw [line width=1.pt] (7.,0.)-- (10.,0.);
\draw [line width=1.pt] (10.,0.)-- (8.5,2.5980762113533165);
\draw [line width=1.pt] (8.5,2.5980762113533165)-- (7.,0.);


\draw (2.8,0.0) node[anchor=north west] {$2$};
\draw (5.8,0.0) node[anchor=north west] {$3$};

\draw (4.3,3) node[anchor=north west] {$1$};
\draw (6.8,0.06) node[anchor=north west] {$c$};
\draw (9.8,0.06) node[anchor=north west] {$b$};
\draw (8.3,2.96) node[anchor=north west] {$a$};

\draw (1.,2.2) node[anchor=north west] {$1\to b$};
\draw (1.,1.8) node[anchor=north west] {$2\to c$};
\draw (1.,1.4) node[anchor=north west] {$3\to a$};

\end{tikzpicture}
\]


Extenderla a una aplicación simplicial entre los complejos indicados en la siguiente figura 

\[
\begin{tikzpicture}[line cap=round,line join=round,>=triangle 45,x=1.0cm,y=1.0cm]
\clip(0.068,-1.) rectangle (12.334666666666664,3);
\draw [line width=1.pt] (3.,0.)-- (6.,0.);
\draw [line width=1.pt] (6.,0.)-- (4.5,2.5980762113533165);
\draw [line width=1.pt] (4.5,2.5980762113533165)-- (3.,0.);
\draw [line width=1.pt] (5.25,1.2990381056766587)-- (3.75,1.2990381056766582);
\draw [line width=1.pt] (3.75,1.2990381056766582)-- (4.5,0.);
\draw [line width=1.pt] (4.5,0.)-- (5.25,1.2990381056766587);
\draw [line width=1.pt] (7.,1.)-- (8.,0.);
\draw [line width=1.pt] (8.,0.)-- (10.,1.);
\draw [line width=1.pt] (10.,1.)-- (10.446666666666665,2.008);
\draw [line width=1.pt] (10.446666666666665,2.008)-- (8.004,2.6266666666666576);
\draw [line width=1.pt] (8.004,2.6266666666666576)-- (7.,1.);
\draw [line width=1.pt] (7.,1.)-- (10.446666666666665,2.008);
\draw [line width=1.pt] (10.446666666666665,2.008)-- (10.,1.);
\draw [line width=1.pt] (10.,1.)-- (7.,1.);
\draw (3.2,1.5706666666666609) node[anchor=north west] {$2$};
\draw (4.3,0.0) node[anchor=north west] {$1$};
\draw (5.38,1.5493333333333275) node[anchor=north west] {$3$};
\draw (6.6,1.1653333333333287) node[anchor=north west] {$c$};
\draw (10.030666666666665,1.2) node[anchor=north west] {$b$};
\draw (10.4,2.2) node[anchor=north west] {$a$};
\end{tikzpicture}
\]
\end{ejercicio}
\begin{solucion}
Vamos primero a etiquetar los vértices añadidos

\[
\begin{tikzpicture}[line cap=round,line join=round,>=triangle 45,x=1.0cm,y=1.0cm]
\clip(0.068,-1.) rectangle (12.334666666666664,3.2);
\draw [line width=1.pt] (3.,0.)-- (6.,0.);
\draw [line width=1.pt] (6.,0.)-- (4.5,2.5980762113533165);
\draw [line width=1.pt] (4.5,2.5980762113533165)-- (3.,0.);
\draw [line width=1.pt] (5.25,1.2990381056766587)-- (3.75,1.2990381056766582);
\draw [line width=1.pt] (3.75,1.2990381056766582)-- (4.5,0.);
\draw [line width=1.pt] (4.5,0.)-- (5.25,1.2990381056766587);
\draw [line width=1.pt] (7.,1.)-- (8.,0.);
\draw [line width=1.pt] (8.,0.)-- (10.,1.);
\draw [line width=1.pt] (10.,1.)-- (10.446666666666665,2.008);
\draw [line width=1.pt] (10.446666666666665,2.008)-- (8.004,2.6266666666666576);
\draw [line width=1.pt] (8.004,2.6266666666666576)-- (7.,1.);
\draw [line width=1.pt] (7.,1.)-- (10.446666666666665,2.008);
\draw [line width=1.pt] (10.446666666666665,2.008)-- (10.,1.);
\draw [line width=1.pt] (10.,1.)-- (7.,1.);
\draw (3.2,1.5706666666666609) node[anchor=north west] {$2$};
\draw (4.3,0.0) node[anchor=north west] {$1$};
\draw (4.3,3) node[anchor=north west] {$4$};
\draw (6.,0) node[anchor=north west] {$5$};
\draw (2.6,0.0) node[anchor=north west] {$6$};
\draw (5.38,1.5493333333333275) node[anchor=north west] {$3$};
\draw (6.6,1.1653333333333287) node[anchor=north west] {$c$};
\draw (10.030666666666665,1.2) node[anchor=north west] {$b$};
\draw (10.4,2.2) node[anchor=north west] {$a$};
\draw (7.8,3.1) node[anchor=north west] {$d$};
\draw (7.8,0) node[anchor=north west] {$e$};
\end{tikzpicture}
\]

A partir de esto definimos la aplicación simplicial no trivial (podríamos enviar todos los nuevos vértices a los que ya tenían preimagen, pero eso no es interesante)
\begin{align*}
&1\to b & 4\to d\\
&2\to c & 5\to b\\
&3\to a & 6\to e
\end{align*}
\end{solucion}

\newpage

\begin{ejercicio}{1.15}
Probar que la imagen de un subcomplejo por una aplicación simplicial es un subcomplejo y que la imagen inversa de un subcomplejo es subcomplejo.
\end{ejercicio}
\begin{solucion}
Sea $\varphi:K_1\to K_2$ una aplicación simplicial y sean $L_1\subseteq K_1$ y $L_2\subseteq K_2$ subcomplejos. En primer lugar es claro $f(L_1)\subseteq K_2$. Sean $\sigma\in f(L_1)$ y consideremos $\tau\leq\sigma$. Supongamos sin pérdida de generalidad que $\sigma=(w_1,\dots, w_s, \dots, w_n)$ y que $\tau=(w_1,\dots, w_s)$. Como $\sigma\in f(L_1)$, existen $v_1,\dots, v_s,\dots, v_n$ tales que $f(v_i)=w_i$. Basta entonces tomar $v_1,\dots, v_s$. Sean $\sigma,\tau\in f(L_1)$ con $\sigma=(w_1,\dots, w_n)$ y $\tau=(w_1',\dots, w_s')$. Entonces existen existen $v_1,\dots, v_n$ y $v_1',\dots, v_s'$ con $f(v_i)=w_i$ y $f(v_j')=w_j'$. Por tanto, basta tomar las preimagenes de los vértices que forman la intersección.

Con $f^{-1}(L_2)\subseteq K_1$ se hace de forma análoga.
\end{solucion}

\newpage

\begin{ejercicio}{1.16}
Probar que una aplicación simplicial puede cubrir todos los vértices y no ser sobreyectiva.
\end{ejercicio}
\begin{solucion}
Basta tomar la inclusión de un triángulo hueco en el triángulo relleno.
\end{solucion}

\newpage

\begin{ejercicio}{1.17}
Sean $|K| = |L| = [0,1]$, teniendo $K$ vértices en $0$, $1/3$ y $1$ y $L$ en $0$, $2/3$ y $1$.
Sea $f(x) = x^2$. Probar que $f$ de $K$ en $L$ no admite aproximación simplicial.

Análogamente, probar que no existe aproximación simplicial de $f$ de $sd K$ en $L$, y encontrar una aproximación simplicial de $f$ de $sd^2 K$ en $L$.
\end{ejercicio}
\begin{solucion}
 Voy a hacer el primero por la definición para que se vea lo duro que es y lo demás lo hacemos usando resultados de teoría. Una aproximación simplicial $\varphi:K\to L$ de $f$ debe cumplir por definición $\varphi(0)=0$ y $\varphi(1)=1$. Además $\varphi(1/3)\in\{0,2/3,1\}$. Como $f(1/3)=1/9\in (0,2/3)$, para que $\varphi(1/3)$ esté en el símplice soporte de $f(1/3)$, debe ser $0$ o $2/3$. Si $\varphi(1/3)=0$, consideramos el punto $0.9\in |K|$. $f(0.9)=0.81\in (2/3,1)$. Expresando 0.9 como suma convexa de los vértices de su soporte en $L$ tenemos que $\lambda=0.6$, de modo que $\varphi(0.9)=\lambda \varphi(1/3)+(1-\lambda)\varphi(1)=0.4\in (0,2/3)$, por lo que $\varphi(0.9)$ no está en el soporte de $f(0.9)$. Si $\varphi(1/3)=2/3$, entonces $f(2/3)$ está en el interior de $(0,2/3)$, mientras que $\varphi(2/3)=0.5\varphi(1/3)+0.5\varphi(1)=0.83$ está en el interior de $(2/3,1)$.

En $sdK$ tenemos los vértices $0,1/6,1/3, 2/3,1$. Se tiene que $f$ admite una aproximación simplicial $\varphi:sdK\to L$ si y solo si para cada $v\in sdK$ existe $w\in L$ de modo que $\mathring{st}(v,sdK)\subseteq f^{-1}(\mathring{st}(w,L))$. Tenemos que $\mathring{st}(1,sdK)=(2/3,1)$, pero $f^{-1}(2/3)>2/3$, luego se da la contención inversa de forma estricta, luego $f$ no admite una aproximación simplicial. 

En el caso de $sd^2K$ se comprueba por inspección que se da el resultado. 

%Nuevamente, $\varphi(0)=0$ y $\varphi(1)=1$. Como $f(\mathring{st}(2/3,sdK))\subseteq \mathring{st}(\varphi(2/3), L)$, $\varphi(2/3)=2/3$. Usando el mismo resultado, como $f(\mathring{st}(1/3,sdK))\subsetneq (0,2/3)$, $\varphi(1/3)\in\{0,2/3\}$. 
\end{solucion}

\end{document}
