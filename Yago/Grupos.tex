\documentclass[twoside, 11pt]{article}
\usepackage{../estilo-apuntes}
%\usepackage{amsmath,amssymb}
%\usepackage[utf8]{inputenc}
%\usepackage[spanish]{babel}
%\usepackage{caption}
%\usepackage[]{graphicx}
%\usepackage{enumerate}
%\usepackage{amsthm}
%\usepackage{tikz-cd}
%\usetikzlibrary{babel}
%\usepackage{pgf,tikz}
%\usepackage{mathrsfs}
%\usepackage{bm}  
%\usetikzlibrary{arrows}
%\usetikzlibrary{cd}
%\usepackage[spanish]{babel}
%\usepackage{fancyhdr}
%\usepackage{titlesec}
%\usepackage{floatrow}
%\usepackage{makeidx}
%\usepackage[tocflat]{tocstyle}
%\usetocstyle{standard}
%\usepackage{subfiles}
%\usepackage{color}  
%\usepackage{hyperref}
%\hypersetup{colorlinks=true,citecolor=red, linkcolor=blue}
%\usepackage{eurosym}
%%\usepackage{ntheorem}
%
%
%\renewcommand{\baselinestretch}{1,4}
%\setlength{\oddsidemargin}{0.25in}
%\setlength{\evensidemargin}{0.25in}
%\setlength{\textwidth}{6in}
%\setlength{\topmargin}{0.1in}
%\setlength{\headheight}{0.1in}
%\setlength{\headsep}{0.1in}
%\setlength{\textheight}{8in}
%\setlength{\footskip}{0.75in}
%
%\theoremstyle{definition}
%
%\newtheorem{teorema}{Teorema}[section]
%\newtheorem{defi}[teorema]{Definición}
%\newtheorem{coro}[teorema]{Corolario}
%\newtheorem{lemma}[teorema]{Lema}
%\newtheorem{ej}[teorema]{Ejemplo}
%\newtheorem{ejs}[teorema]{Ejemplos}
%\newtheorem{observacion}[teorema]{Observación}
%\newtheorem{observaciones}[teorema]{Observaciones}
%\newtheorem{prop}[teorema]{Proposición}
%\newtheorem{propi}[teorema]{Propiedades}
%\newtheorem{nota}[teorema]{Nota}
%\newtheorem{notas}[teorema]{Notas}
%\newtheorem*{dem}{Demostración}
%\newtheorem{ejer}[teorema]{Ejercicio}
%\newtheorem{problem}[teorema]{Problema}
%\newtheorem{concl}[teorema]{Conclusión}
%
%\providecommand{\abs}[1]{\lvert#1\rvert}
%\providecommand{\sen}[1]{sen #1}
%\providecommand{\norm}[1]{\lVert#1\rVert}
%\providecommand{\ninf}[1]{\norm{#1}_\infty}
%\providecommand{\numn}[1]{\norm{#1}_1}
%\providecommand{\gabs}[1]{\left|{#1}\right|}
%\newcommand{\bor}[1]{\mathcal{B}(#1)}
%\newcommand{\R}{\mathbb{R}}
%\newcommand{\Z}{\mathbb{Z}}
%\newcommand{\N}{\mathbb{N}}
%\newcommand{\Q}{\mathbb{Q}}
%\newcommand{\C}{\mathbb{C}}
%\newcommand{\Pro}{\mathbb{P}}
%\newcommand{\Tau}{\mathcal{T}}
%\newcommand{\verteq}{\rotatebox{90}{$\,=$}}
%\newcommand{\vertequiv}{\rotatebox{110}{$\,\equiv$}}
%\providecommand{\lrg}{\longrightarrow}
%\providecommand{\func}[2]{\colon{#1}\longrightarrow{#2}}
%\newcommand*{\QED}{\hfill\ensuremath{\blacksquare}}
%\newcommand*\circled[1]{\tikz[baseline=(char.base)]{
%            \node[shape=circle,draw,inner sep=1.5pt] (char) {#1};}}
%\newcommand*{\longhookarrow}{\ensuremath{\lhook\joinrel\relbar\joinrel\rightarrow}}


\begin{document}
%\title{Topología de Superficies}
%\author{Antonio Rafael Quintero Toscano\\ Javier Aguilar Martín}
%\date{Curso 2016/2017}
%\maketitle

\author{Javier Aguilar Martín }
\date{\today}
\title{Teoría Geométrica de Grupos}

\maketitle


\begin{abstract}
Estos apuntes están basados en las clases de Yago Antolín en la Escuela JAE de 2018. Sus notas originales pueden ser encontradas en el siguiente enlace: \url{https://sites.google.com/site/yagoanpi/curso-escuela-jae}
\end{abstract}


	\vfill
	Esta obra está licenciada bajo la Licencia Creative Commons Atribución 3.0 España. Para ver una copia de esta licencia, visite \url{http://creativecommons.org/licenses/by/3.0/es/} o envíe una carta a Creative Commons, PO Box 1866, Mountain View, CA 94042, USA.


\newpage
\tableofcontents

\newpage

\section{Grupos finitamente generados como espacios métricos}
\subsection{Grafos de Cayley}
\begin{teorema}[Cayley]
Todo subgrupo es isomorfo a un subgrupo de un grupo simétrico.
\end{teorema}
\begin{dem}
Dado un grupo $G$ definimos $\sigma:G\to Biy(G)=Sym_{|G|}$ como $x\mapsto \sigma_x:G\to G$, siendo $\sigma_x(g)=gx^{-1}$. Tenemos que $\sigma$ es un homomorfismo pues
$$
\sigma_{xy}(g)=g(xy)^{-1}=gy^{-1}x^{-1}=(gy^{-1})x^{-1})=\sigma_y(g)x^{-1}=\sigma_x(\sigma_y(g)).
$$
Además es inyectiva pues 
$$
\sigma_x(g)=g\Leftrightarrow gx^{-1}=g\Leftrightarrow x^{-1}=1_G\Leftrightarrow x=1_G.
$$
\QED
\end{dem}

\begin{ej}
Vamos a cómo se interpreta la demostración anterior en un ejemplo concreto. Sea $G=(\Z/3\Z)^3$ y sean $x=(1,0,0)$, $y=(0,1,)$, $z=(0,0,1)$. Tenemos 
\[
\sigma_x\equiv (000,\ 100)(010,\ 110)(001,\ 101)(011,\ 111)
\]
expresada como producto de ciclos en los que los grupos de tres números se corresponden a su vez a ciclos. Análogamente se definirían $\sigma_y$ y $\sigma_z$. Podemos interpretar geométricamente estas aplicaciones como la acción que intercambia los vértices de un cubo como en la figura siguiente:
DIBUJO
\end{ej}

\begin{defi}
Dado un grupo $G$ y un conjunto $X\subseteq G$ definimos el \textbf{grafo de Cayley} de $G$ respecto a $X$ como el grafo $\Gamma(G,X)$ con vértices $V(\Gamma)=G$ y aristas $E(\Gamma)=G\times X$, interpretadas como una arista entre $g$ y $gx$, habitualmente etiquetada con $x$. DIBUJO
\end{defi}

\begin{observaciones}\
\begin{enumerate}
\item Por comodida hemos cambiado $g\xrightarrow{\sigma_x}gx^{-1}$ por $g\xrightarrow{x}gx$. Si hubiésemos hecho la prueba del teorema de Cayley con la acción por la derecha no habría sido necesario el inverso.
\item $\Gamma(G,X)$ es conexo si y solo si $\gene{X}=G$.
\item En cada arista de $\Gamma(G,X)$ entra y sale exactamente una arista por cada elemento de $G$. En particular $\Gamma(G,X)$ recubre el grafo con vértices $G$ y ciclos por cada elemento de $x$ en todos los vértices. DIBUJO
\item La acción de $G$ en $G=V(\Gamma)$ por la izquierda induce una acción de $G$ en $\Gamma(G,X)$ por automorfismos de grafos etiquetados
\[
g\cdot (a\xrightarrow{x}ax)=ga\xrightarrow{x}gax
\]
\end{enumerate}
\end{observaciones}

\begin{ej}
$\Z^2=\gene{a,b\mid ab=ba}=\{a^nb^m\mid n,m\in\Z\}$. DIBUJO
\end{ej}

\begin{defi}
Una \textbf{palabra} $w$ en $X\cup X^{-1}$ (o por simplicidad en $X$) es una sucesión fintia de elementos de $X\cup X^{-1}$.
\end{defi}

Dado un conjunto $Y$, $Y^*$ es el monoide generado por $Y$, formado por las palabras en $Y$ con la operación de concatenación.

\begin{defi}
Un \textbf{camino} $p$ en un grafo $\Gamma$ es una sucesión $v_0,e_1^{\varepsilon_1}, v_1,e_2^{\varepsilon_2},\dots, e_n^{\varepsilon_n},v_n$, donde $v_i\in V(\Gamma)$, $e_i\in E(\Gamma)$, $\varepsilon_i\in\{\pm 1\}$, de modo que si $\varepsilon_i=1$ entonces la arista es de la forma $v_{i-1}\xrightarrow{e_i}v_{i}$ y si $\varepsilon_i=-1$ la arista es de la forma $v_{i-1}\xleftarrow{e_i}v_i$. 

En particular, dado $p$ de $v_0$ a $v_n$ podemos definir el camino $p^{-1}\equiv v_n,e_n^{-\varepsilon_n}, \dots, e_1^{-\varepsilon_1},v_0$ que va de $v_n$ a $v_0$. 
\end{defi}

Una palabra $w$ en $X\cup X^{-1}$ codifica un único camino $p$ en $\Gamma(G,X)$ (módulo vértice inicial) de longitud $l(p)$ igual a $l(w)$, la longitud de la palabra. 

\begin{ej}
En $\Z^2$ consideramos la palabra $w=ababa^{-1}ab^{-2}=_G aa$. DIBUJOS

Si dos palabras empiezan y acaban en los mismos vértices entonces describen una relación en el grupo. 
\end{ej}
  
\subsection{Grupos como espacios métricos}

Sea $G$ un grupo finitamente generado y $X$ un conjunto finito de generadores. Definimos
\[
d_X:G\times G\to\Z_{\geq 0}
\]
\[
(g,h)\mapsto\min\{l(p)\mid p\text{ es un camino de }g\text{ a }h\}
\]
\begin{prop}
La aplicación $d_X$ es una métrica.
\end{prop}
\begin{dem}\
\begin{itemize}
\item Trivialmente $d_X\geq 0$ porque los caminos tienen longitud no negativa y $d_X(g,h)=0\Leftrightarrow g=h$ pues el camino constante tiene longitud 0.
\item Se tiene $d_X(g,h)=d_X(h,g)$ tomando el camino inverso.
\item Por último, $d_X(h,g)\leq d_X(h,k)+d_X(k,g)$ al ser la concatenación de los caminos de $h$ a $k$ y de $k$ a $g$ un camino válido aunque no sea el que dé necesariamente la longitud mínima. 
\end{itemize}
\QED
\end{dem}

La acción de $G \curvearrowright G$ por la izquierda preserva la métrica, esto es, para todo $a,g,h\in G$, $d_X(ag,ah)=d_X(g,h)$. En efecto, si $gx_1\cdots x_n=h$, entonces $agx_1\cdots x_n=ah$. En particular, $d_X(g,h)=d_X(1_G,g^{-1}h)$. Definimos pues, $|g|_X=d(1_G,g)$. 

\begin{observacion}
La métrica no es única, sino que depende de los generadores. DIBUJO
\end{observacion}

\begin{defi}
Sean $(A,d_A)$ y $(B,d_B)$ espacios métricos. Sean $\lambda\geq 1$ y $c\geq 0$. Una función $f:A\to B$ es un \textbf{encaje $(\lambda,c)$-quasi-isométrico} si
\[
\frac{1}{\lambda}d_A(x,y)-c\leq d_B(f(x),f(y))\leq \lambda d_A(x,y)+c.
\]
Diremos que $f$ es una \textbf{$(\lambda,c)$-quasi-isometría} si además para todo $b\in B$ existe $a\in A$ tal que $d_B(f(a),b)\leq c$ (a esta propiedad se la conoce como \textbf{quasi-sobreyectividad}). Cuando no sea necesario especificar los parámetros hablaremos de encajes quasi-isométricos y quasi-isometrías. En caso de existir una quasi-isometría entre $A$ y $B$ diremos que $A$ es quasi-isométrico a $B$.
\end{defi}

\begin{ejs}\
\begin{enumerate}
\item $\Z\hookrightarrow\R$ es un encaje $(1,0)$-quasi-isométrico y una $(1,1)$-quasi-isometría para la distancia usual.
\item $f:\R\to\Z:x\mapsto\lfloor x\rfloor$ es una $(1,1)$-quasi-isometría.
\item 
\end{enumerate}
\end{ejs}

\begin{observacion}
Los encajes quasi-isométricos son funciones de Lipschitz y por tanto continuas.
\end{observacion}

\begin{prop}\
\begin{enumerate}
\item Si $f:A\to B$ y $g:B\to C$ son quasi-isometrías entonces $g\circ f:A\to C$ también lo es.
\item Si $f:A\to B$ es una quasi-isometría, entonces existe una quasi-isometría $g:B\to A$ y una constante $k\geq 0$ tales que
\begin{align*}
d_A(a,g\circ f(a))\leq k \ \forall a\in A\\
d_B(b,f\circ g(b))\leq k \ \forall b\in B
\end{align*}
\end{enumerate}
\end{prop}
\begin{dem}
HACER
\QED
\end{dem}

\begin{coro}
Ser quasi-isométricos es una relación de equivalencia de espacios métricos.
\end{coro}

\begin{lemma}
Sea $G$ un grupo finitamente generado y supongamos que $X\subseteq G$ e $Y\subseteq G$ son dos conjuntos finitos de generadores. Entonces
\[
Id:(G,d_X)\to (G,d_Y)
\]
es una quasi-isometría.
\end{lemma}
\begin{proof}
Sea $M_x=\max_{x\in X} |x|_Y$. Entonces, para todo $x\in X$, existe $w_x\in Y^*$ con $l(w_x)\leq M$. Ahora, 
\begin{align*}
d_X(a,b)=n &\Rightarrow a^{-1}b=x_1\cdots x_n, x_i\in X\\
& \Rightarrow a^{-1}b=w_{x_1}\cdots w_{x_n}\in Y^*\\
& \Rightarrow d_Y(a,b)\leq M_x\cdot d_X(a,b).
\end{align*}
Similarlmente, $d_X(a,b)\leq M_Y d_Y(a,b)$, por lo que
\[
\frac{1}{M_Y}d_X(a,b)\leq d_Y(a,b)\leq M_xd(a,b)
\]
\end{proof}

\begin{coro}
El grafo de Cayley de un grupo finitamente generado es único salvo quasi-isometrías.
\end{coro}

\begin{defi}
Un espacio métrico $(M,d)$ se llama \textbf{geodésico} si para todo $x,y\in M$ existe $p:[0,d(x,y)]\to M$ continua con $p(0)=x$, $p(d(x,y))=y$ y $d(p(t),p(t'))=|t-t'|$ para todo $t,t'\in [0,d(x,y)]$. 
\end{defi}

\begin{ejs}\
\begin{enumerate}
\item $\R^n$ es un espacio métrico geodésico con la métrica euclídea.
\item Un grafo $\Gamma$ puede dotarse de estructura de espacio métrico geodésico identificando las aristas con copias del intervalo $[0,1]$. 
\item El espacio hiperbólico $\mathbb{H}^n$. 
\item Cualquier variedad Riemanniana completa.
\item $\R^n-\{0\}$ no es geodésico. 
\end{enumerate}
\end{ejs}

\begin{defi}
Una acción $G\curvearrowright (M,d)$ es:
\begin{itemize}
\item \textbf{por isometrías} si $d(x,y)=d(gx,gy)$ para todo $x,y\in M$ y para todo $g\in G$. 
\item \textbf{métricamente propia} si para todo $x\in M$ y para todo $r>0$ el conjunto $\{g\in G\mid d(x,gx)<r\}$ es finito.
\item \textbf{co-acotada} si existe $R$ tal que $\bigcup_{g\in G} gB(x,R)=M$ para algún $x\in M$ (esto implica que se cumple para todo $x\in M$).
\end{itemize}
\end{defi}

\begin{lemma}[\v Svarc-Milnor]
Supongamos que $G$ actúa por isometrías en un espacio métrico geodésico $(M,d)$ y la acción es métricamente propia y co-acotada. Entonces $G$ es finitamente generado y quasi-isométrico a $M$. 
\end{lemma}
\begin{dem}
ES MU LARGA
\QED
\end{dem}
\begin{ej}
Si $G$ actúa sobre un grafo localmente finito con un número finito de $G$-órbitas, la acción es en un espacio métrico geodésico y co-acotada, por lo que se verifica el lema. REPASAR ESTO PARA VER SI LO ENTIENDO BIEN
\end{ej}
\begin{coro}
Si $G$ es finitamente generado y $H\leq G$ es un subgrupo de índice finito, entonces $H$ es finitamente generado y quasi-isométrico a $G$. 
\end{coro}
\begin{dem}
CREO QUE ESTÁ EN LOS APUNTES 
\QED
\end{dem}

\section{Invariantes de quasi-isométría: Crecimiento}
Con las herramientas de esta sección podremos responder de forma sencilla a la pregunta de si $\Z$ es quasi-isométrico a $\Z^2$.
\subsection{Función de crecimiento}
\begin{defi}
Dado un grupo finitamente generado $G$ y un conjunto finito de generadores $X\subseteq G$, definimos
\begin{enumerate}[a)]
\item la \textbf{función de crecimiento por bolas} 
\[
\beta_{(G,X)}(n)=\#\{g\in G\mid |g|_X\leq n\}
\]
\item la \textbf{función de crecimiento por esferas}
\[
\sigma_{(G,X)}(n)=\#\{g\in G\mid |g|_X=n\}
\]
\end{enumerate}
\end{defi}

\begin{observacion}
$\sigma_{(G,X)}(n)=\beta_{(G,X)}(n)-\beta_{(G,X)}(n-1)$
\end{observacion}

\begin{ejs}\
\begin{enumerate}
\item $G=\Z$, $X=\{1\}$: $\beta(n)=1+2n$, $\sigma(n)=\begin{cases}
1 & n=0\\
2 & n>0
\end{cases}$
\item $G=\Z^2$, $X=\{(1,0),(0,1)\}$: $\beta(n)=1+2n(n+1)$, $\sigma(n)=4n$. 
\item GRUPO DEL FAROLERO
\end{enumerate}
\end{ejs}

\begin{observacion}
$\beta(n)\leq (2|X|)^n$
\end{observacion}

\begin{defi}
Diremos que $\alpha:\R_{\geq 0}\to\R_{\geq }$ es \textbf{de crecimiento} si es no decreciente. Si $\alpha_1$ y $\alpha_2$ son funciones de crecimiento, diremos que $\alpha_1$ \textbf{domina débilmente} a $\alpha_2$ ($\alpha_2\prec \alpha_1$) si existen $\lambda\geq 1$ y $c\geq 0$ tales que
\[
\alpha_2(n)\leq \lambda\alpha_1(\lambda n+c)+c, \forall n\in\R_{\geq 0}.
\]
Si $\alpha_1\prec\alpha_2$ y $\alpha_2\prec\alpha_1$ decimos que $\alpha_1$ y $\alpha_2$ son \textbf{débilmente equivalentes} y escribimos $\alpha_1\sim\alpha_2$.
\end{defi}

\begin{lemma}
La relación $\sim$ es de equivalencia.
\end{lemma}
La demostración es trivial, así que se deja como ejercicio.

\begin{ejs}\
\begin{enumerate}
\item Sean $a,b\in\R_{\geq 0}$, entonces $\alpha_1(n)=n^a\prec n^n=\alpha_2(n)\Leftrightarrow a\leq b$ PROBARLO
\item $e^{an}\sim a^{bn}$ PROBARLO
\end{enumerate}
\end{ejs}

\begin{lemma}
Si $(G,X)\sim_{QI} (H,Y)$ entonces $\beta_{(G,X)}\sim\beta_{(H,Y)}$.
\end{lemma}
\begin{proof}
Sea $f:(G,d_X)\to (H,d_Y)$ una $(\lambda,c)$-quasi-isometría, es decir,
\[
\frac{1}{\lambda}d_X(a,b)-c\leq d_Y(f(a),f(b))\leq \lambda d_X(a,b)+c.
\]
Entonces, $f(B_X(1_G,n))\subseteq B_Y(f(1_G),\lambda n+c)$ y $|f^{-1}(b)|\leq \beta_{(G,X)}(\lambda c)=k$. Así que $\beta_{(G,X)}(n)\leq k\beta_{(H,Y)}(\lambda n+c)$, luego $\beta_{(G,X)}\prec \beta_{(H,Y)}$. De forma simétrica obtenemos $\beta_{(H,Y)}\prec \beta_{(G,X)}$.
\end{proof}

\begin{coro}
$\Z\not\sim_{QI}\Z^2$.
\end{coro}

Sea $G=\gene{X}$ con $X$ finito. Escribimos $\beta_G$ para denotar una función equivalente a $\beta_{(G,X)}$.

\begin{defi}
Diremos que $G$ es de crecimiento \textbf{exponencial} si $\beta_G\sim 2^n$ y que es de crecimiento \textbf{polinomial} si $\beta_G\prec cn^d$. En este segundo caso definimos $d(G)=\inf\{s\mid \exists c:\beta_{(G,X)}(n)\leq cn^s\}$. Por último, se dirá que $G$ es de crecimiento \textbf{subexponencial} o \textbf{intermedio} si no es exponencial ni polinomial. 
\end{defi}
Grigorchuk probó en 1981 que para todo $0<\alpha_1<\alpha_2<1$ existe un grupo $G$ con $2^{n\alpha_1}\prec \beta_G\prec 2^{n\alpha_2}$, es decir, que existen grupos de crecimiento subexponencial. 

\begin{prop}
Sean $H\leq G$ ambos finitamente generados y sea $N\trianglelefteq G$.
\begin{enumerate}
\item Si $[G:H]<\infty$, entonces $\beta_G\sim\beta_H$.
\item Si $G$ es de crecimiento polinómico entonces también lo son $H$ y $G/N$. Además, $d(H),d(G/N)\leq d(G)$.
\item Si $G$ es de crecimiento polinómico y $[G:H]=\infty$, entonces $d(H)\leq d(G)-1$.
\end{enumerate}
\end{prop}
\begin{dem}\
\begin{enumerate}
\item Si $[G:H]<\infty$, entonces $G\sim_{QI} H$, luego $\beta_G\sim\beta_H$.
\item Sean $Y\subseteq H$ y $X\subseteq G$ conjuntos finitos de generadores respectivamente. Definimos $M_Y=\max_{y\in Y}|y|_X$. Entonces, $\beta_{(H,Y)}(n)\leq \beta_{(G,X)}(M_Y n)$, con lo que $H$ es de crecimiento polinómico.

Sea ahora $\overline{X}=\{xN\mid x\in X\}$, entonces $G/N=\gene{\overline{X}}$ y $\beta_{(G,X)}(n)\geq \beta_{(G/N,\overline{X})}(n)$. 

\item Sean $Y\subseteq X\subseteq G$ finitos tales que $G=\gene{X}$ y $H=\gene{Y}$. Sean $Hg_1,\dots, Hg_n$ distintas clases laterales de $H$. Pongamos $K=Hg_1\cup\cdots\cup Hg_n$. Por ser $H$ de índice infinito, existe $x\in X\cup X^{-1}$ tal que $Kx\neq K$, ya que si no $K=G$. Por tanto, existe $i$ tal que $Hg_ix$ es una clase lateral nueva. Empezando con $\{H\}$, para todo $n$ podemos encontrar $Hg_1,\dots, Hg_n$ con $|g_i|_X\leq i$. Entonces $\beta_{(G,X)}(n)\leq n\beta_{(H,Y)}(n)$.\QED
\end{enumerate}
\end{dem}

\begin{teorema}
Sea $G$ un grupo infinito finitamente generado. Las siguientes afirmaciones son equivalentes:
\begin{enumerate}
\item $G$ contiene un subgrupo cíclico infinito de índice finito.
\item $G\sim_{QI}\Z$.
\item $\beta_G\sim n$.
\end{enumerate}
\end{teorema}
\begin{dem}
Las implicaciones $1\Rightarrow 2\Rightarrow 3$ ya están probadas. Para probar $3\Rightarrow 1$ necesitamos encontrar un elemento de orden infinito $a$, ya que entonces, si $[G:\gene{a}]=\infty$, la proposición anterior nos daría $\beta_G(n)>n^2$, lo que sería una contradicción.

Sea $X$ un conjunto finito de generadores. Ponemos un orden (alfabético) en $X\cup X^{-1}$ y lo extendemos a $(X\cup X^{-1})^*$ de forma long-lex, es decir, $w<w'\Leftrightarrow l(w)<l(w')$ o $l(w)=l(w')$ pero $w$ precede alfabéticamente a $w'$. 

Para todo $g\in G$, existe un único $w_g=\min_{long-lex}\{w\in(X\cup X^{-1})^*\mid w=_G g\}$. Llamaremos a estos $w_g$ \textbf{palabras distinguidas} y observamos que toda subpalabra de una palabra distinguida es distinguida. Llamamos a una palabra $w=x_1\cdots x_n$ \textbf{$p$-periódica} si $x_i=x_{i+p}$ para $1\leq i\leq n-p$. 

Necesitamos el siguiente lema cuya demostración posponemos.

\begin{lemma}
Si una palabra $p$-periódica contiene una subpalabra $q$-periódica de longitud al menos $p+q-1$, entonces $w$ es $d$-periódica con $d=\gcd(p,q)$.
\end{lemma}

Como $\beta_{(G,X)}\leq cn$, existe $m$ tal que el número de elementos $g\in G$ con $|g|_X=m$ es menor que $m$, ya que de lo contrario $\beta_{(G,X)}(n)\leq \sum_{i=1}^n i\sim n^2$.

Vamos a probar el siguiente enunciado:

``Dado $g\in G$ con $|g|_X\geq 2m$, existen palabras $u_g, s_g,v_g$ tal que $l(u_g),l(v_g)\leq m$, $s_g$ $p$-periódica con $p\leq m$ y $w_g=u_gs_gv_g$.''

Observamos que este enunciado implica la existencia de elementos de órdenes arbitrariamente grandes, ya que $l(s_g)\geq |g|_X-2m$ y podemos escribir $s_g=t_g^ks_g'$, donde $t_g\in G$ con $|t_g|_X=p\leq m$ y $t_g$ tiene orden mayor que $k$, ya que si fuera menor o igual, podríamos reducir $s_g$ y $w_g$ no sería de longitud mínima. Como $w_g$ puede tener longitud arbitraria y una cantidad finita de $t_g$ para elegir, debe de existir alguno de orden infinito. 

Si $w_g=y_1\cdots y_n$, $n\geq 2m$, ponemos $v_i=y_i\cdots y_{i+m-1}$. Tenemos que $l(v_i)=m$ y como $n\geq 2m$ existen $v_1,\dots, v_{m+1}$. Como no hay más de $m$ palabras distinguidas de longitud $m$, existen $i,j$ tales que $v_i=v_j$ y tenemos que $y_i\cdots y_j\cdots y_{j+m-1}$ es $p$-periódica con $p\leq j-i\leq m$.

Ahora ponemos $w_g=vsu$ y asumimos que $s=y_i\cdots y_j$ tiene longitud máxima siendo $p$-periódica con $p\leq m$ ($p$ mínimo). Tenemos que probar que $l(v)\leq m$ y $l(u)\leq m$. Ambos casos son análogos así que probamos el primero. Supongamos que $l(v)>m$, esto es, $i-1>m$. Vamos a obtener una contradicción. 

Como $i-1>m-p$ y $l(s)\geq m+p$, existen $v_{i+p-m-1},\dots, v_{i+p-1}$ subpalabras de longitud $m$ y por el razonamiento de antes existen dos iguales, digamos $v_k$ y $v_l$. Tenemos $s=y_i\cdots y_j$ $p$-periódica, $p\leq m$, $l(s)\geq m+p$; $b=y_k\cdots y_l\cdots y_{l+m-1}$ $q$-periódica, $q\leq k-1\leq m$, $l(q)=m+q$ (asumimos $q$ mínimo). 

\begin{itemize}
\item si $i<k$: $s$ y $b$ intersecan en $y_i\cdots y_{k+m+q-1}$ de longitud $(k+q+m-1)-(i-1)\geq p+q-1$. Por tanto, por el lema: si $p<q$, $q$ no es periodo mínimo de $b$; si $q<p$, $p$ no es periodo mínimo de $s$; si $p=q$, $s$ no es de longitud máxima.
\item si $i\geq k$: $s$ y $b$ intersecan en $y_k\cdots y_l\cdots y_{l+m-1}$ de longitud $q\leq l-k\leq l-i<p$, así que por el lema $p$ no es periodo mínimo de $s$. 
\end{itemize}

\QED
\end{dem}

\begin{proof}[Demostración del lema]
Sea $w=y_1\cdots y_n$ con $y_i=y_{i+p}$ y $u=y_i\cdots y_j$ con $y_k=y_{k+q}$ y $j-i+1\geq p+q-1$. Se tiene que $p=tq+r$ con $0\leq r<q$. Si $r=0$, 
\[
y_{i-1}=y_{i-1+p}=y_{i-1+p+(t+1)q}=y_{i-1+q}
\]
por lo que $y_{i-1}\cdots y_j$ es $q$-periódica. Repitiendo obtenemos que $w$ es $q$-periódica, luego de hecho tiene periodo $\gcd(p,q)$. 

Si $r>0$, para $i\leq k\leq j-p$, $y_k=y_{k+p}=y_{k+p-tq}=y_{k+r}$, así que $y_i\cdots y_{j-p+r}$ es una subpalabra $r$-periódica de $y_i\cdots y_j$ de longitud $j-p+r-1+1\geq q+r-1$ y por inducción $y_i\cdots y_j$ tiene periodo $\gcd(q,r)=\gcd(p,q)|p$, por lo que estamos en el caso $r=0$. 
\end{proof}

\section{Invariantes de quasi-isométría: Finales}
Hemos visto que $\Z\not\sim_{QI}\Z^2$ usando el crecimiento. Ahora usaremos otro invariante. Para ello introduciremos una variante de los caminos que comporte bien con las quasi-isometrías. 
\subsection{$k$-rayos}
\begin{defi}
Dado una grafo $\Gamma$ y $k\geq 1$:
\begin{itemize}
\item Un \textbf{$k$-camino} es una sucesión de vértices $\{v_i\}_{i=1}^n$ tal que $d_\Gamma(v_i,v_{i+1})\leq k$. En particular, los caminos son $1$-caminos.
\item Un \textbf{$k$-rayo propio} es una sucesión infinita de vértices $\gamma=\{v_i\}_{i\in\N}$ tal que para todo $v\in \Gamma$ y para todo $r\geq 0$, existen $N_\gamma$, $v$ y $R$ tales que para todo $i\geq N_\gamma$, $v_i\notin B(v,R)$.  
\end{itemize}
Dos $k$-rayos propios $\alpha=\{a_i\}_{i\in \N}$ y $\beta=\{b_i\}_{i\in\N}$ se dicen equivalentes si para todo $v\in\Gamma$ y para algún $k\geq 1$ existe un $k$-camino $\{v_i\}_{i=1}^n$ tal que $v_i\notin B(v,R)$, $v_0=a_j$ con $j\geq N_\alpha$ y $v_n=b_l$ con $l\geq N_\beta$. REVISAR ESTA DEFINICIÓN CON LOS APUNTES POR EL K
\end{defi}
\begin{ejs}
DIBUJOS
\end{ejs}

\begin{prop}
La relación entre $k$-rayos propios de ser equivalentes es una relación de equivalencia.
\end{prop}
\begin{dem}
DEMOSTRAR 
\QED
\end{dem}

\begin{observacion}
Si $k\leq k'$ entonces todo $k$-rayo propio es un $k'$-rayo propio.  Si $\alpha$ y $\beta$ son $k$-rayos propios equivalentes, entonces también son equivalentes como $k'$-rayos propios. Así, tenemos que la aplicación $\{ k$-rayos propios$\}\longrightarrow\{k'$-rayos propios$\}$ es inyectiva y que $\{ k$-rayos propios$\}/\sim\longrightarrow\{k'$-rayos propios$\}/sim$ está bien definida.
\end{observacion}

\begin{prop}
Si $k\leq k'$, entonces $\{ k$-rayos propios$\}/\sim\longrightarrow\{k'$-rayos propios$\}/sim$ es una biyección. 
\end{prop}
\begin{dem}
DEMOSTRAR
\QED
\end{dem}

\subsection{Finales}

\begin{defi}
Se definen los \textbf{finales} de un grafo $\Gamma$ como $Ends(\Gamma)=\{ k$-rayos propios$\}/\sim$, que está bien definido salvo biyección por las observaciones de la sección anterior. El número de finales se denota $|Ends(\Gamma)|$.
\end{defi}

\begin{lemma}
Si $f:\Gamma\to\Delta$ es una $(\lambda,c)$-quasi-isometría, entonces la aplicación $\{ k$-rayos propios en $\Gamma\}\longrightarrow\{(\lambda k+c)$-rayos propios en $\Delta\}$ dada por $\alpha=\{a_i\}\mapsto f(\alpha)=\{f(a_i)\}$ induce una biyección $Ends(\Gamma)\to Ends(\Delta)$. 
\end{lemma}

MIRAR SI ESTÁ LA DEMO EN LOS APUNTES Y SI NO INTENTARLA

\begin{coro}
Si $G$ es finitamente generado, entonces el número de finales es independiente de los generadores y los denotaremos por $|Ends(G)|$.
\end{coro}

\begin{coro}
$|Ends(G)|\neq |Ends(H)|\Rightarrow G\not\sim_{QI} H$.
\end{coro}

\begin{ejs}\
\begin{enumerate}
\item $2=|Ends(\Z)|\neq |Ends(\Z^2)|=1\Rightarrow \Z\not\sim_{QI}\Z^2$.
\item Si $G$ es finito, entonces $|Ends(G)|=0$.
\end{enumerate}
\end{ejs}

\begin{observaciones}\
\begin{enumerate}
\item Si $\Gamma=\Gamma(G,X)$ es un grafo de Cayley y $\alpha=\{a_i\}_{i=1}^\infty$ es un $k$-rayo propio, entonces $g\alpha=\{ga_i\}_{i=1}^\infty$ es un $k$-rayo propio, pues la distancia se preserva por multiplicación a la izquierda. 
\item Si $\alpha,\beta$ son $k$-rayos propios y $\gamma$ es un $k$-camino que los conecta, entonces $g\alpha,g\beta$ y $g\gamma$ tienen la misma propiedad.  Por tanto, la acción de $G$ preserva la equivalencia de $k$-rayos y tenemos  $G\curvearrowright Ends(\Gamma)$. 
\end{enumerate}
\end{observaciones}

\begin{ej}
Sea $G=\{f:\Z\to\Z\mid f(x)=ax+b, a=\pm 1, b\in\Z\}=Isom(\Z)$. Podemos identificar $f$ con el par $(a,b)$, con lo que la composición tendría la forma $(a,b)\circ (a',b')=(aa',ab'+b)$. Se tiene que $G=\gene{(-1,0), (1,1)}=\gene{\color{blue}{r} ,\color{red}{t}}$, pues $(-1,b)=(-1,0)(1,1)^{-b}$ y $(1,b)=(1,1)^b$. 
DIBUJO (REVISAR LAS FLECHAS)

Denotamos por $t^\infty$ al camino empezando en $(1,0)$ y siempre siguiendo aristas en orientación positiva. Análogamente definimos $t^{-\infty}$ con orientación negativa. 

Tenemos entonces que $Ends(\Gamma(G,X))=\{[t^\infty],[t^{-\infty}]\}$. Además, REVISAR LA ACCIÓN

Obsérvese que $G\cong \Z\rtimes\Z/2\Z\cong D_\infty$.
\end{ej}

\begin{teorema}[Arzela-Ascoli]
Sea $A$ un espacio métrico compacto, $B$ un espacio métrico separable y $f_n:B\to A$ una sucesión de funciones equicontinuas. Entonces existe una subsucesión que converge a una función continua $f:B\to A$.
\end{teorema}

\begin{coro}
Si $\Gamma$ es un grafo localmente finito, toda sucesión $\{\rho_n\}$ de caminos geodésicos que tengan un vértice común $v$ y $l(p_n)\to\infty$ cuando $n\to\infty$ tiene una subsucesión que converge a un camino geodésico infinito. REVISAR ESTE ENUNCIADO CON LOS APUNTES
\end{coro}

\begin{prop}
Si $G$ es finitamente generado, entonces $|Ends(G)|\in\{0,1,2,\infty\}$.
\end{prop}
\begin{dem}
COPIAR
\QED
\end{dem}

\section{Grupos finitamente presentados}
\subsection{Grupos libres}
\subsection{Transformaciones de Tietze}
\subsection{El problema de la palabra}

\section{Lenguas formales}
\subsection{Autómatas finitos (Finite State Automata)}
\subsection{Push-down Automata}

\section{Miscelánea. Para saber más}

\end{document}