\documentclass[twoside, 11pt]{article}
\usepackage{../estilo-apuntes}
%\usepackage{amsmath,amssymb}
%\usepackage[utf8]{inputenc}
%\usepackage[spanish]{babel}
%\usepackage{caption}
%\usepackage[]{graphicx}
%\usepackage{enumerate}
%\usepackage{amsthm}
%\usepackage{tikz-cd}
%\usetikzlibrary{babel}
%\usepackage{pgf,tikz}
%\usepackage{mathrsfs}
%\usepackage{bm}  
%\usetikzlibrary{arrows}
%\usetikzlibrary{cd}
%\usepackage[spanish]{babel}
%\usepackage{fancyhdr}
%\usepackage{titlesec}
%\usepackage{floatrow}
%\usepackage{makeidx}
%\usepackage[tocflat]{tocstyle}
%\usetocstyle{standard}
%\usepackage{subfiles}
%\usepackage{color}  
%\usepackage{hyperref}
%\hypersetup{colorlinks=true,citecolor=red, linkcolor=blue}
%\usepackage{eurosym}
%%\usepackage{ntheorem}
%
%
%\renewcommand{\baselinestretch}{1,4}
%\setlength{\oddsidemargin}{0.25in}
%\setlength{\evensidemargin}{0.25in}
%\setlength{\textwidth}{6in}
%\setlength{\topmargin}{0.1in}
%\setlength{\headheight}{0.1in}
%\setlength{\headsep}{0.1in}
%\setlength{\textheight}{8in}
%\setlength{\footskip}{0.75in}
%
%\theoremstyle{definition}
%
%\newtheorem{teorema}{Teorema}[section]
%\newtheorem{defi}[teorema]{Definición}
%\newtheorem{coro}[teorema]{Corolario}
%\newtheorem{lemma}[teorema]{Lema}
%\newtheorem{ej}[teorema]{Ejemplo}
%\newtheorem{ejs}[teorema]{Ejemplos}
%\newtheorem{observacion}[teorema]{Observación}
%\newtheorem{observaciones}[teorema]{Observaciones}
%\newtheorem{prop}[teorema]{Proposición}
%\newtheorem{propi}[teorema]{Propiedades}
%\newtheorem{nota}[teorema]{Nota}
%\newtheorem{notas}[teorema]{Notas}
%\newtheorem*{dem}{Demostración}
%\newtheorem{ejer}[teorema]{Ejercicio}
%\newtheorem{problem}[teorema]{Problema}
%\newtheorem{concl}[teorema]{Conclusión}
%
%\providecommand{\abs}[1]{\lvert#1\rvert}
%\providecommand{\sen}[1]{sen #1}
%\providecommand{\norm}[1]{\lVert#1\rVert}
%\providecommand{\ninf}[1]{\norm{#1}_\infty}
%\providecommand{\numn}[1]{\norm{#1}_1}
%\providecommand{\gabs}[1]{\left|{#1}\right|}
%\newcommand{\bor}[1]{\mathcal{B}(#1)}
%\newcommand{\R}{\mathbb{R}}
%\newcommand{\Z}{\mathbb{Z}}
%\newcommand{\N}{\mathbb{N}}
%\newcommand{\Q}{\mathbb{Q}}
%\newcommand{\C}{\mathbb{C}}
%\newcommand{\Pro}{\mathbb{P}}
%\newcommand{\Tau}{\mathcal{T}}
%\newcommand{\verteq}{\rotatebox{90}{$\,=$}}
%\newcommand{\vertequiv}{\rotatebox{110}{$\,\equiv$}}
%\providecommand{\lrg}{\longrightarrow}
%\providecommand{\func}[2]{\colon{#1}\longrightarrow{#2}}
%\newcommand*{\QED}{\hfill\ensuremath{\blacksquare}}
%\newcommand*\circled[1]{\tikz[baseline=(char.base)]{
%            \node[shape=circle,draw,inner sep=1.5pt] (char) {#1};}}
%\newcommand*{\longhookarrow}{\ensuremath{\lhook\joinrel\relbar\joinrel\rightarrow}}


\begin{document}
%\title{Topología de Superficies}
%\author{Antonio Rafael Quintero Toscano\\ Javier Aguilar Martín}
%\date{Curso 2016/2017}
%\maketitle

\author{Javier Aguilar Martín }
\date{\today}
\title{Teoría Geométrica de Grupos}

\maketitle


\begin{abstract}
Estos apuntes están basados en las clases de Yago Antolín en la Escuela JAE de 2018. Sus notas originales pueden ser encontradas en el siguiente enlace: \url{https://sites.google.com/site/yagoanpi/curso-escuela-jae}
\end{abstract}


	\vfill
	Esta obra está licenciada bajo la Licencia Creative Commons Atribución 3.0 España. Para ver una copia de esta licencia, visite \url{http://creativecommons.org/licenses/by/3.0/es/} o envíe una carta a Creative Commons, PO Box 1866, Mountain View, CA 94042, USA.


\newpage
\tableofcontents

\newpage

\section{Grupos finitamente generados como espacios métricos}
\subsection{Grafos de Cayley}
\begin{teorema}[Cayley]
Todo subgrupo es isomorfo a un subgrupo de un grupo simétrico.
\end{teorema}
\begin{dem}
Dado un grupo $G$ definimos $\sigma:G\to Biy(G)=Sym_{|G|}$ como $x\mapsto \sigma_x:G\to G$, siendo $\sigma_x(g)=gx^{-1}$. Tenemos que $\sigma$ es un homomorfismo pues
$$
\sigma_{xy}(g)=g(xy)^{-1}=gy^{-1}x^{-1}=(gy^{-1})x^{-1})=\sigma_y(g)x^{-1}=\sigma_x(\sigma_y(g)).
$$
Además es inyectiva pues 
$$
\sigma_x(g)=g\Leftrightarrow gx^{-1}=g\Leftrightarrow x^{-1}=1_G\Leftrightarrow x=1_G.
$$
\QED
\end{dem}

\begin{ej}
Vamos a cómo se interpreta la demostración anterior en un ejemplo concreto. Sea $G=(\Z/3\Z)^3$ y sean $x=(1,0,0)$, $y=(0,1,)$, $z=(0,0,1)$. Tenemos 
\[
\sigma_x\equiv (000,\ 100)(010,\ 110)(001,\ 101)(011,\ 111)
\]
expresada como producto de ciclos en los que los grupos de tres números se corresponden a su vez a ciclos. Análogamente se definirían $\sigma_y$ y $\sigma_z$. Podemos interpretar geométricamente estas aplicaciones como la acción que intercambia los vértices de un cubo como en la figura siguiente:
DIBUJO
\end{ej}

\begin{defi}
Dado un grupo $G$ y un conjunto $X\subseteq G$ definimos el \textbf{grafo de Cayley} de $G$ respecto a $X$ como el grafo $\Gamma(G,X)$ con vértices $V(\Gamma)=G$ y aristas $E(\Gamma)=G\times X$, interpretadas como una arista entre $g$ y $gx$, habitualmente etiquetada con $x$. DIBUJO
\end{defi}

\begin{observaciones}\
\begin{enumerate}
\item Por comodida hemos cambiado $g\xrightarrow{\sigma_x}gx^{-1}$ por $g\xrightarrow{x}gx$. Si hubiésemos hecho la prueba del teorema de Cayley con la acción por la derecha no habría sido necesario el inverso.
\item $\Gamma(G,X)$ es conexo si y solo si $\gene{X}=G$.
\item En cada arista de $\Gamma(G,X)$ entra y sale exactamente una arista por cada elemento de $G$. En particular $\Gamma(G,X)$ recubre el grafo con vértices $G$ y ciclos por cada elemento de $x$ en todos los vértices. DIBUJO
\item La acción de $G$ en $G=V(\Gamma)$ por la izquierda induce una acción de $G$ en $\Gamma(G,X)$ por automorfismos de grafos etiquetados
\[
g\cdot (a\xrightarrow{x}ax)=ga\xrightarrow{x}gax
\]
\end{enumerate}
\end{observaciones}

\begin{ej}
$\Z^2=\gene{a,b\mid ab=ba}=\{a^nb^m\mid n,m\in\Z\}$. DIBUJO
\end{ej}

\begin{defi}
Una \textbf{palabra} $w$ en $X\cup X^{-1}$ (o por simplicidad en $X$) es una sucesión fintia de elementos de $X\cup X^{-1}$.
\end{defi}

Dado un conjunto $Y$, $Y^*$ es el monoide generado por $Y$, formado por las palabras en $Y$ con la operación de concatenación.

\begin{defi}
Un \textbf{camino} $p$ en un grafo $\Gamma$ es una sucesión $v_0,e_1^{\varepsilon_1}, v_1,e_2^{\varepsilon_2},\dots, e_n^{\varepsilon_n},v_n$, donde $v_i\in V(\Gamma)$, $e_i\in E(\Gamma)$, $\varepsilon_i\in\{\pm 1\}$, de modo que si $\varepsilon_i=1$ entonces la arista es de la forma $v_{i-1}\xrightarrow{e_i}v_{i}$ y si $\varepsilon_i=-1$ la arista es de la forma $v_{i-1}\xleftarrow{e_i}v_i$. 

En particular, dado $p$ de $v_0$ a $v_n$ podemos definir el camino $p^{-1}\equiv v_n,e_n^{-\varepsilon_n}, \dots, e_1^{-\varepsilon_1},v_0$ que va de $v_n$ a $v_0$. 
\end{defi}

Una palabra $w$ en $X\cup X^{-1}$ codifica un único camino $p$ en $\Gamma(G,X)$ (módulo vértice inicial) de longitud $l(p)$ igual a $l(w)$, la longitud de la palabra. 

\begin{ej}
En $\Z^2$ consideramos la palabra $w=ababa^{-1}ab^{-2}=_G aa$. DIBUJOS

Si dos palabras empiezan y acaban en los mismos vértices entonces describen una relación en el grupo. 
\end{ej}
  
\subsection{Grupos como espacios métricos}

Sea $G$ un grupo finitamente generado y $X$ un conjunto finito de generadores. Definimos
\[
d_X:G\times G\to\Z_{\geq 0}
\]
\[
(g,h)\mapsto\min\{l(p)\mid p\text{ es un camino de }g\text{ a }h\}
\]
\begin{prop}
La aplicación $d_X$ es una métrica.
\end{prop}
\begin{dem}\
\begin{itemize}
\item Trivialmente $d_X\geq 0$ porque los caminos tienen longitud no negativa y $d_X(g,h)=0\Leftrightarrow g=h$ pues el camino constante tiene longitud 0.
\item Se tiene $d_X(g,h)=d_X(h,g)$ tomando el camino inverso.
\item Por último, $d_X(h,g)\leq d_X(h,k)+d_X(k,g)$ al ser la concatenación de los caminos de $h$ a $k$ y de $k$ a $g$ un camino válido aunque no sea el que dé necesariamente la longitud mínima. 
\end{itemize}
\QED
\end{dem}

La acción de $G \curvearrowright G$ por la izquierda preserva la métrica, esto es, para todo $a,g,h\in G$, $d_X(ag,ah)=d_X(g,h)$. En efecto, si $gx_1\cdots x_n=h$, entonces $agx_1\cdots x_n=ah$. En particular, $d_X(g,h)=d_X(1_G,g^{-1}h)$. Definimos pues, $|g|_X=d(1_G,g)$. 

\begin{observacion}
La métrica no es única, sino que depende de los generadores. DIBUJO
\end{observacion}

\begin{defi}
Sean $(A,d_A)$ y $(B,d_B)$ espacios métricos. Sean $\lambda\geq 1$ y $c\geq 0$. Una función $f:A\to B$ es un \textbf{encaje $(\lambda,c)$-quasi-isométrico} si
\[
\frac{1}{\lambda}d_A(x,y)-c\leq d_B(f(x),f(y))\leq \lambda d_A(x,y)+c.
\]
Diremos que $f$ es una \textbf{$(\lambda,c)$-quasi-isometría} si además para todo $b\in B$ existe $a\in A$ tal que $d_B(f(a),b)\leq c$ (a esta propiedad se la conoce como \textbf{quasi-sobreyectividad}). Cuando no sea necesario especificar los parámetros hablaremos de encajes quasi-isométricos y quasi-isometrías. En caso de existir una quasi-isometría entre $A$ y $B$ diremos que $A$ es quasi-isométrico a $B$.
\end{defi}

\begin{ejs}\
\begin{enumerate}
\item $\Z\hookrightarrow\R$ es un encaje $(1,0)$-quasi-isométrico y una $(1,1)$-quasi-isometría para la distancia usual.
\item $f:\R\to\Z:x\mapsto\lfloor x\rfloor$ es una $(1,1)$-quasi-isometría.
\item 
\end{enumerate}
\end{ejs}

\begin{observacion}
Los encajes quasi-isométricos son funciones de Lipschitz y por tanto continuas.
\end{observacion}

\begin{prop}\
\begin{enumerate}
\item Si $f:A\to B$ y $g:B\to C$ son quasi-isometrías entonces $g\circ f:A\to C$ también lo es.
\item Si $f:A\to B$ es una quasi-isometría, entonces existe una quasi-isometría $g:B\to A$ y una constante $k\geq 0$ tales que
\begin{align*}
d_A(a,g\circ f(a))\leq k \ \forall a\in A\\
d_B(b,f\circ g(b))\leq k \ \forall b\in B
\end{align*}
\end{enumerate}
\end{prop}
\begin{dem}
HACER
\QED
\end{dem}

\begin{coro}
Ser quasi-isométricos es una relación de equivalencia de espacios métricos.
\end{coro}

\begin{lemma}
Sea $G$ un grupo finitamente generado y supongamos que $X\subseteq G$ e $Y\subseteq G$ son dos conjuntos finitos de generadores. Entonces
\[
Id:(G,d_X)\to (G,d_Y)
\]
es una quasi-isometría.
\end{lemma}
\begin{proof}
Sea $M_x=\max_{x\in X} |x|_Y$. Entonces, para todo $x\in X$, existe $w_x\in Y^*$ con $l(w_x)\leq M$. Ahora, 
\begin{align*}
d_X(a,b)=n &\Rightarrow a^{-1}b=x_1\cdots x_n, x_i\in X\\
& \Rightarrow a^{-1}b=w_{x_1}\cdots w_{x_n}\in Y^*\\
& \Rightarrow d_Y(a,b)\leq M_x\cdot d_X(a,b).
\end{align*}
Similarlmente, $d_X(a,b)\leq M_Y d_Y(a,b)$, por lo que
\[
\frac{1}{M_Y}d_X(a,b)\leq d_Y(a,b)\leq M_xd(a,b)
\]
\end{proof}

\begin{coro}
El grafo de Cayley de un grupo finitamente generado es único salvo quasi-isometrías.
\end{coro}

\begin{defi}
Un espacio métrico $(M,d)$ se llama \textbf{geodésico} si para todo $x,y\in M$ existe $p:[0,d(x,y)]\to M$ continua con $p(0)=x$, $p(d(x,y))=y$ y $d(p(t),p(t'))=|t-t'|$ para todo $t,t'\in [0,d(x,y)]$. 
\end{defi}

\begin{ejs}\
\begin{enumerate}
\item $\R^n$ es un espacio métrico geodésico con la métrica euclídea.
\item Un grafo $\Gamma$ puede dotarse de estructura de espacio métrico geodésico identificando las aristas con copias del intervalo $[0,1]$. 
\item El espacio hiperbólico $\mathbb{H}^n$. 
\item Cualquier variedad Riemanniana completa.
\item $\R^n-\{0\}$ no es geodésico. 
\end{enumerate}
\end{ejs}

\begin{defi}
Una acción $G\curvearrowright (M,d)$ es:
\begin{itemize}
\item \textbf{por isometrías} si $d(x,y)=d(gx,gy)$ para todo $x,y\in M$ y para todo $g\in G$. 
\item \textbf{métricamente propia} si para todo $x\in M$ y para todo $r>0$ el conjunto $\{g\in G\mid d(x,gx)<r\}$ es finito.
\item \textbf{co-acotada} si existe $R$ tal que $\bigcup_{g\in G} gB(x,R)=M$ para algún $x\in M$ (esto implica que se cumple para todo $x\in M$).
\end{itemize}
\end{defi}

\begin{lemma}[\v Svarc-Milnor]
Supongamos que $G$ actúa por isometrías en un espacio métrico geodésico $(M,d)$ y la acción es métricamente propia y co-acotada. Entonces $G$ es finitamente generado y quasi-isométrico a $M$. 
\end{lemma}
\begin{dem}
ES MU LARGA
\QED
\end{dem}
\begin{ej}
Si $G$ actúa sobre un grafo localmente finito con un número finito de $G$-órbitas, la acción es en un espacio métrico geodésico y co-acotada, por lo que se verifica el lema. REPASAR ESTO PARA VER SI LO ENTIENDO BIEN
\end{ej}
\begin{coro}
Si $G$ es finitamente generado y $H\leq G$ es un subgrupo de índice finito, entonces $H$ es finitamente generado y quasi-isométrico a $G$. 
\end{coro}
\begin{dem}
CREO QUE ESTÁ EN LOS APUNTES 
\QED
\end{dem}

\section{Invariantes de quasi-isométría: Crecimiento}
Con las herramientas de esta sección podremos responder de forma sencilla a la pregunta de si $\Z$ es quasi-isométrico a $\Z^2$.
\subsection{Función de crecimiento}
\begin{defi}
Dado un grupo finitamente generado $G$ y un conjunto finito de generadores $X\subseteq G$, definimos
\begin{enumerate}[a)]
\item la \textbf{función de crecimiento por bolas} 
\[
\beta_{(G,X)}(n)=\#\{g\in G\mid |g|_X\leq n\}
\]
\item la \textbf{función de crecimiento por esferas}
\[
\sigma_{(G,X)}(n)=\#\{g\in G\mid |g|_X=n\}
\]
\end{enumerate}
\end{defi}

\begin{observacion}
$\sigma_{(G,X)}(n)=\beta_{(G,X)}(n)-\beta_{(G,X)}(n-1)$
\end{observacion}

\begin{ejs}\
\begin{enumerate}
\item $G=\Z$, $X=\{1\}$: $\beta(n)=1+2n$, $\sigma(n)=\begin{cases}
1 & n=0\\
2 & n>0
\end{cases}$
\item $G=\Z^2$, $X=\{(1,0),(0,1)\}$: $\beta(n)=1+2n(n+1)$, $\sigma(n)=4n$. 
\item GRUPO DEL FAROLERO
\end{enumerate}
\end{ejs}

\begin{observacion}
$\beta(n)\leq (2|X|)^n$
\end{observacion}

\begin{defi}
Diremos que $\alpha:\R_{\geq 0}\to\R_{\geq }$ es \textbf{de crecimiento} si es no decreciente. Si $\alpha_1$ y $\alpha_2$ son funciones de crecimiento, diremos que $\alpha_1$ \textbf{domina débilmente} a $\alpha_2$ ($\alpha_2\prec \alpha_1$) si existen $\lambda\geq 1$ y $c\geq 0$ tales que
\[
\alpha_2(n)\leq \lambda\alpha_1(\lambda n+c)+c, \forall n\in\R_{\geq 0}.
\]
Si $\alpha_1\prec\alpha_2$ y $\alpha_2\prec\alpha_1$ decimos que $\alpha_1$ y $\alpha_2$ son \textbf{débilmente equivalentes} y escribimos $\alpha_1\sim\alpha_2$.
\end{defi}

\begin{lemma}
La relación $\sim$ es de equivalencia.
\end{lemma}
La demostración es trivial, así que se deja como ejercicio.

\begin{ejs}\
\begin{enumerate}
\item Sean $a,b\in\R_{\geq 0}$, entonces $\alpha_1(n)=n^a\prec n^n=\alpha_2(n)\Leftrightarrow a\leq b$ PROBARLO
\item $e^{an}\sim a^{bn}$ PROBARLO
\end{enumerate}
\end{ejs}

\begin{lemma}
Si $(G,X)\sim_{QI} (H,Y)$ entonces $\beta_{(G,X)}\sim\beta_{(H,Y)}$.
\end{lemma}
\begin{proof}
Sea $f:(G,d_X)\to (H,d_Y)$ una $(\lambda,c)$-quasi-isometría, es decir,
\[
\frac{1}{\lambda}d_X(a,b)-c\leq d_Y(f(a),f(b))\leq \lambda d_X(a,b)+c.
\]
Entonces, $f(B_X(1_G,n))\subseteq B_Y(f(1_G),\lambda n+c)$ y $|f^{-1}(b)|\leq \beta_{(G,X)}(\lambda c)=k$. Así que $\beta_{(G,X)}(n)\leq k\beta_{(H,Y)}(\lambda n+c)$, luego $\beta_{(G,X)}\prec \beta_{(H,Y)}$. De forma simétrica obtenemos $\beta_{(H,Y)}\prec \beta_{(G,X)}$.
\end{proof}

\begin{coro}
$\Z\not\sim_{QI}\Z^2$.
\end{coro}

Sea $G=\gene{X}$ con $X$ finito. Escribimos $\beta_G$ para denotar una función equivalente a $\beta_{(G,X)}$.

\begin{defi}
Diremos que $G$ es de crecimiento \textbf{exponencial} si $\beta_G\sim 2^n$ y que es de crecimiento \textbf{polinomial} si $\beta_G\prec cn^d$. En este segundo caso definimos $d(G)=\inf\{s\mid \exists c:\beta_{(G,X)}(n)\leq cn^s\}$. Por último, se dirá que $G$ es de crecimiento \textbf{subexponencial} o \textbf{intermedio} si no es exponencial ni polinomial. 
\end{defi}
Grigorchuk probó en 1981 que para todo $0<\alpha_1<\alpha_2<1$ existe un grupo $G$ con $2^{n\alpha_1}\prec \beta_G\prec 2^{n\alpha_2}$, es decir, que existen grupos de crecimiento subexponencial. 


\section{Invariantes de quasi-isométría: Finales}
\subsection{$k$-rayos}
\subsection{Finales}

\section{Grupos finitamente presentados}
\subsection{Grupos libres}
\subsection{Transformaciones de Tietze}
\subsection{El problema de la palabra}

\section{Lenguas formales}
\subsection{Autómatas finitos (Finite State Automata)}
\subsection{Push-down Automata}

\section{Miscelánea. Para saber más}

\end{document}