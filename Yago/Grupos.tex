\documentclass[twoside, 11pt]{article}
\usepackage{../estilo-apuntes}
%\usepackage{amsmath,amssymb}
%\usepackage[utf8]{inputenc}
%\usepackage[spanish]{babel}
%\usepackage{caption}
%\usepackage[]{graphicx}
%\usepackage{enumerate}
%\usepackage{amsthm}
%\usepackage{tikz-cd}
%\usetikzlibrary{babel}
%\usepackage{pgf,tikz}
%\usepackage{mathrsfs}
%\usepackage{bm}  
%\usetikzlibrary{arrows}
%\usetikzlibrary{cd}
%\usepackage[spanish]{babel}
%\usepackage{fancyhdr}
%\usepackage{titlesec}
%\usepackage{floatrow}
%\usepackage{makeidx}
%\usepackage[tocflat]{tocstyle}
%\usetocstyle{standard}
%\usepackage{subfiles}
%\usepackage{color}  
%\usepackage{hyperref}
%\hypersetup{colorlinks=true,citecolor=red, linkcolor=blue}
%\usepackage{eurosym}
%%\usepackage{ntheorem}
%
%
%\renewcommand{\baselinestretch}{1,4}
%\setlength{\oddsidemargin}{0.25in}
%\setlength{\evensidemargin}{0.25in}
%\setlength{\textwidth}{6in}
%\setlength{\topmargin}{0.1in}
%\setlength{\headheight}{0.1in}
%\setlength{\headsep}{0.1in}
%\setlength{\textheight}{8in}
%\setlength{\footskip}{0.75in}
%
%\theoremstyle{definition}
%
%\newtheorem{teorema}{Teorema}[section]
%\newtheorem{defi}[teorema]{Definición}
%\newtheorem{coro}[teorema]{Corolario}
%\newtheorem{lemma}[teorema]{Lema}
%\newtheorem{ej}[teorema]{Ejemplo}
%\newtheorem{ejs}[teorema]{Ejemplos}
%\newtheorem{observacion}[teorema]{Observación}
%\newtheorem{observaciones}[teorema]{Observaciones}
%\newtheorem{prop}[teorema]{Proposición}
%\newtheorem{propi}[teorema]{Propiedades}
%\newtheorem{nota}[teorema]{Nota}
%\newtheorem{notas}[teorema]{Notas}
%\newtheorem*{dem}{Demostración}
%\newtheorem{ejer}[teorema]{Ejercicio}
%\newtheorem{problem}[teorema]{Problema}
%\newtheorem{concl}[teorema]{Conclusión}
%
%\providecommand{\abs}[1]{\lvert#1\rvert}
%\providecommand{\sen}[1]{sen #1}
%\providecommand{\norm}[1]{\lVert#1\rVert}
%\providecommand{\ninf}[1]{\norm{#1}_\infty}
%\providecommand{\numn}[1]{\norm{#1}_1}
%\providecommand{\gabs}[1]{\left|{#1}\right|}
%\newcommand{\bor}[1]{\mathcal{B}(#1)}
%\newcommand{\R}{\mathbb{R}}
%\newcommand{\Z}{\mathbb{Z}}
%\newcommand{\N}{\mathbb{N}}
%\newcommand{\Q}{\mathbb{Q}}
%\newcommand{\C}{\mathbb{C}}
%\newcommand{\Pro}{\mathbb{P}}
%\newcommand{\Tau}{\mathcal{T}}
%\newcommand{\verteq}{\rotatebox{90}{$\,=$}}
%\newcommand{\vertequiv}{\rotatebox{110}{$\,\equiv$}}
%\providecommand{\lrg}{\longrightarrow}
%\providecommand{\func}[2]{\colon{#1}\longrightarrow{#2}}
%\newcommand*{\QED}{\hfill\ensuremath{\blacksquare}}
%\newcommand*\circled[1]{\tikz[baseline=(char.base)]{
%            \node[shape=circle,draw,inner sep=1.5pt] (char) {#1};}}
%\newcommand*{\longhookarrow}{\ensuremath{\lhook\joinrel\relbar\joinrel\rightarrow}}


\begin{document}
%\title{Topología de Superficies}
%\author{Antonio Rafael Quintero Toscano\\ Javier Aguilar Martín}
%\date{Curso 2016/2017}
%\maketitle

\author{Javier Aguilar Martín }
\date{\today}
\title{Teoría Geométrica de Grupos}

\maketitle


\begin{abstract}
Estos apuntes están basados en las clases de Yago Antolín en la Escuela JAE de 2018. Sus notas originales pueden ser encontradas en el siguiente enlace: \url{https://sites.google.com/site/yagoanpi/curso-escuela-jae}
\end{abstract}


	\vfill
	Esta obra está licenciada bajo la Licencia Creative Commons Atribución 3.0 España. Para ver una copia de esta licencia, visite \url{http://creativecommons.org/licenses/by/3.0/es/} o envíe una carta a Creative Commons, PO Box 1866, Mountain View, CA 94042, USA.


\newpage
\tableofcontents

\newpage

\section{Grupos finitamente generados como espacios métricos}
\subsection{Grafos de Cayley}
\begin{teorema}[Cayley]
Todo subgrupo es isomorfo a un subgrupo de un grupo simétrico.
\end{teorema}
\begin{dem}
Dado un grupo $G$ definimos $\sigma:G\to Biy(G)=Sym_{|G|}$ como $x\mapsto \sigma_x:G\to G$, siendo $\sigma_x(g)=gx^{-1}$. Tenemos que $\sigma$ es un homomorfismo pues
$$
\sigma_{xy}(g)=g(xy)^{-1}=gy^{-1}x^{-1}=(gy^{-1})x^{-1})=\sigma_y(g)x^{-1}=\sigma_x(\sigma_y(g)).
$$
Además es inyectiva pues 
$$
\sigma_x(g)=g\Leftrightarrow gx^{-1}=g\Leftrightarrow x^{-1}=1_G\Leftrightarrow x=1_G.
$$
\QED
\end{dem}

\begin{ej}
Vamos a cómo se interpreta la demostración anterior en un ejemplo concreto. Sea $G=(\Z/3\Z)^3$ y sean $x=(1,0,0)$, $y=(0,1,)$, $z=(0,0,1)$. Tenemos 
\[
\sigma_x\equiv (000,\ 100)(010,\ 110)(001,\ 101)(011,\ 111)
\]
expresada como producto de ciclos en los que los grupos de tres números se corresponden a su vez a ciclos. Análogamente se definirían $\sigma_y$ y $\sigma_z$. Podemos interpretar geométricamente estas aplicaciones como la acción que intercambia los vértices de un cubo como en la figura siguiente:
DIBUJO
\end{ej}

\begin{defi}
Dado un grupo $G$ y un conjunto $X\subseteq G$ definimos el \textbf{grafo de Cayley} de $G$ respecto a $X$ como el grafo $\Gamma(G,X)$ con vértices $V(\Gamma)=G$ y aristas $E(\Gamma)=G\times X$, interpretadas como una arista entre $g$ y $gx$, habitualmente etiquetada con $x$. DIBUJO
\end{defi}

\begin{observaciones}\
\begin{enumerate}
\item 
\item $\Gamma(G,X)$ es conexo si y solo si $\gene{X}=G$.
\item En cada arista de $\Gamma(G,X)$ entra y sale exactamente una arista por cada elemento de $G$. 
\item 
\end{enumerate}
\end{observaciones}

\subsection{Grupos como espacios métricos}


\section{Invariantes de quasi-isométría: Crecimiento}
\subsection{Función de crecimiento}

\section{Invariantes de quasi-isométría: Finales}
\subsection{$k$-rayos}
\subsection{Finales}

\section{Grupos finitamente presentados}
\subsection{Grupos libres}
\subsection{Transformaciones de Tietze}
\subsection{El problema de la palabra}

\section{Lenguas formales}
\subsection{Autómatas finitos (Finite State Automata)}
\subsection{Push-down Automata}

\section{Miscelánea. Para saber más}

\end{document}