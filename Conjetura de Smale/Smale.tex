\documentclass[twoside, 11pt]{article}
\usepackage{../estilo-apuntes}
%\usepackage{amsmath,amssymb}
%\usepackage[utf8]{inputenc}
%\usepackage[spanish]{babel}
%\usepackage{caption}
%\usepackage[]{graphicx}
%\usepackage{enumerate}
%\usepackage{amsthm}
%\usepackage{tikz-cd}
%\usetikzlibrary{babel}
%\usepackage{pgf,tikz}
%\usepackage{mathrsfs}
%\usepackage{bm}  
%\usetikzlibrary{arrows}
%\usetikzlibrary{cd}
%\usepackage[spanish]{babel}
%\usepackage{fancyhdr}
%\usepackage{titlesec}
%\usepackage{floatrow}
%\usepackage{makeidx}
%\usepackage[tocflat]{tocstyle}
%\usetocstyle{standard}
%\usepackage{subfiles}
%\usepackage{color}  
%\usepackage{hyperref}
%\hypersetup{colorlinks=true,citecolor=red, linkcolor=blue}
%\usepackage{eurosym}
%%\usepackage{ntheorem}
%
%
%\renewcommand{\baselinestretch}{1,4}
%\setlength{\oddsidemargin}{0.25in}
%\setlength{\evensidemargin}{0.25in}
%\setlength{\textwidth}{6in}
%\setlength{\topmargin}{0.1in}
%\setlength{\headheight}{0.1in}
%\setlength{\headsep}{0.1in}
%\setlength{\textheight}{8in}
%\setlength{\footskip}{0.75in}
%
%\theoremstyle{definition}
%
%\newtheorem{teorema}{Teorema}[section]
%\newtheorem{defi}[teorema]{Definición}
%\newtheorem{coro}[teorema]{Corolario}
%\newtheorem{lemma}[teorema]{Lema}
%\newtheorem{ej}[teorema]{Ejemplo}
%\newtheorem{ejs}[teorema]{Ejemplos}
%\newtheorem{observacion}[teorema]{Observación}
%\newtheorem{observaciones}[teorema]{Observaciones}
%\newtheorem{prop}[teorema]{Proposición}
%\newtheorem{propi}[teorema]{Propiedades}
%\newtheorem{nota}[teorema]{Nota}
%\newtheorem{notas}[teorema]{Notas}
%\newtheorem*{dem}{Demostración}
%\newtheorem{ejer}[teorema]{Ejercicio}
%\newtheorem{problem}[teorema]{Problema}
%\newtheorem{concl}[teorema]{Conclusión}
%
%\providecommand{\abs}[1]{\lvert#1\rvert}
%\providecommand{\sen}[1]{sen #1}
%\providecommand{\norm}[1]{\lVert#1\rVert}
%\providecommand{\ninf}[1]{\norm{#1}_\infty}
%\providecommand{\numn}[1]{\norm{#1}_1}
%\providecommand{\gabs}[1]{\left|{#1}\right|}
%\newcommand{\bor}[1]{\mathcal{B}(#1)}
%\newcommand{\R}{\mathbb{R}}
%\newcommand{\Z}{\mathbb{Z}}
%\newcommand{\N}{\mathbb{N}}
%\newcommand{\Q}{\mathbb{Q}}
%\newcommand{\C}{\mathbb{C}}
%\newcommand{\Pro}{\mathbb{P}}
%\newcommand{\Tau}{\mathcal{T}}
%\newcommand{\verteq}{\rotatebox{90}{$\,=$}}
%\newcommand{\vertequiv}{\rotatebox{110}{$\,\equiv$}}
%\providecommand{\lrg}{\longrightarrow}
%\providecommand{\func}[2]{\colon{#1}\longrightarrow{#2}}
%\newcommand*{\QED}{\hfill\ensuremath{\blacksquare}}
%\newcommand*\circled[1]{\tikz[baseline=(char.base)]{
%            \node[shape=circle,draw,inner sep=1.5pt] (char) {#1};}}
%\newcommand*{\longhookarrow}{\ensuremath{\lhook\joinrel\relbar\joinrel\rightarrow}}


\begin{document}
%\title{Topología de Superficies}
%\author{Antonio Rafael Quintero Toscano\\ Javier Aguilar Martín}
%\date{Curso 2016/2017}
%\maketitle

\author{Javier Aguilar Martín }
\date{\today}
\title{La conjetura de Smale:\\ $Diff(S^3)\simeq O(4)$}

\maketitle


\begin{abstract}
Un guion que no pretende ser totalmente detallado ni riguroso sobre la prueba de Hatcher de la conjetura de Smale. RECUERDA PONER LAS DOS COSAS COMO BIBLIOGRAFÍA
\end{abstract}


	\vfill
	Esta obra está licenciada bajo la Licencia Creative Commons Atribución 3.0 España. Para ver una copia de esta licencia, visite \url{http://creativecommons.org/licenses/by/3.0/es/} o envíe una carta a Creative Commons, PO Box 1866, Mountain View, CA 94042, USA.


\newpage
\tableofcontents

\newpage

\section{Introducción y enunciado equivalente}
\subsection{Introduccion}


\subsection{Enunciado equivalente}
Es conocido que $Diff(S^n)\simeq O(n+1)\times Diff(D^n\ rel\ \partial D^n)$ para todo $n$. Por tanto, la conjetura de Smale $Diff(S^4)\simeq O(4)$ es equivalente a $Diff(D^3\ rel\ \partial D^3)\simeq *$. El teorema puede ser reenunciado como como que para los espacios de smooth embeddings $Emb(D^3,\R^3)$ y $Emb(S^2,\R^n)$, la restricción $\rho:Emb(D^3,\R^3)\to Emb(S^2,\R^n)$ induce sobreyección en $\pi_k$ para todo $k$, pues $\rho$ es una fibración cuyas fibra es precisamente $Diff(D^3\ rel\ \partial D^3)$ (se usa la sucesión exacta larga de homotopía de una fibración). Consideremos ahora el diagrama
\[
\begin{tikzcd}
Emb(D^3,\R^3)\arrow[rr, "\rho"] \arrow[dr, "\simeq"]& & Emb(S^2,\R^n)\arrow[dl]\\
&GL(n,\R)&
\end{tikzcd}
\]
donde la equivalencia de homotopía consiste en evaluar la derivada en un punto. A partir de esto podemos enunciar equivalentemente el teorema como ``el espacio de smoothly embedded 2-esferas en $\R^3$ es contráctil''. Este espacio es el espacio de órbitas $Emb(S^2,\R^3)/Diff(S^2)$. Como $Diff(S^2)\simeq O(3)$, la contractibilidad del espacio es equivalente a que $\rho$ sea equivalencia de homotopía, pues $GL(n,\R)\simeq O(n)$ mediante Gram-Schmidt. Esta última versión será la que se probará.


\section{Normalización, partición de la unidad y cirugía}
\subsection{Normalización}
\begin{defi}
Una superficie orientada $S$ de $\R^3$ se dice \emph{elemental} si verifica las siguientes condiciones:
\begin{enumerate}
\item Cada curva del borde está en un plano horizontal, siendo $S$ vertical en un entorno de cada curva. 
\item La intersección de $S$ con una línea vertical es conexa (posiblemente vacía).
\item Si denotamos por $S^+$ (respectivamente $S^-$) a la adherencia del conjunto de puntos donde la normal tiene coordenaza $z$ estrictamente positiva (respectivamente negativa), $S^+$ y $S^-$ están aisladas la una de la otra. 
\end{enumerate}
\end{defi}
\subsection{Partición de la unidad}
\subsection{Cirugía}

\section{Contornos}
\subsection{Compatibilidad con la suma}
\subsection{Compatibilidad con la diferencia}

\section{Cómo hace continua la familia $\Sigma_{to}$}
\subsection{Orden en las caras}
\subsection{Conclusión}


\section{Prueba de la conjetura de Smale}



\end{document}