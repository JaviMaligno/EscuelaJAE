\documentclass[twoside, 11pt]{article}
\usepackage{../estilo-apuntes}
%\usepackage{amsmath,amssymb}
%\usepackage[utf8]{inputenc}
%\usepackage[spanish]{babel}
%\usepackage{caption}
%\usepackage[]{graphicx}
%\usepackage{enumerate}
%\usepackage{amsthm}
%\usepackage{tikz-cd}
%\usetikzlibrary{babel}
%\usepackage{pgf,tikz}
%\usepackage{mathrsfs}
%\usepackage{bm}  
%\usetikzlibrary{arrows}
%\usetikzlibrary{cd}
%\usepackage[spanish]{babel}
%\usepackage{fancyhdr}
%\usepackage{titlesec}
%\usepackage{floatrow}
%\usepackage{makeidx}
%\usepackage[tocflat]{tocstyle}
%\usetocstyle{standard}
%\usepackage{subfiles}
%\usepackage{color}  
%\usepackage{hyperref}
%\hypersetup{colorlinks=true,citecolor=red, linkcolor=blue}
%\usepackage{eurosym}
%%\usepackage{ntheorem}
%
%
%\renewcommand{\baselinestretch}{1,4}
%\setlength{\oddsidemargin}{0.25in}
%\setlength{\evensidemargin}{0.25in}
%\setlength{\textwidth}{6in}
%\setlength{\topmargin}{0.1in}
%\setlength{\headheight}{0.1in}
%\setlength{\headsep}{0.1in}
%\setlength{\textheight}{8in}
%\setlength{\footskip}{0.75in}
%
%\theoremstyle{definition}
%
%\newtheorem{teorema}{Teorema}[section]
%\newtheorem{defi}[teorema]{Definición}
%\newtheorem{coro}[teorema]{Corolario}
%\newtheorem{lemma}[teorema]{Lema}
%\newtheorem{ej}[teorema]{Ejemplo}
%\newtheorem{ejs}[teorema]{Ejemplos}
%\newtheorem{observacion}[teorema]{Observación}
%\newtheorem{observaciones}[teorema]{Observaciones}
%\newtheorem{prop}[teorema]{Proposición}
%\newtheorem{propi}[teorema]{Propiedades}
%\newtheorem{nota}[teorema]{Nota}
%\newtheorem{notas}[teorema]{Notas}
%\newtheorem*{dem}{Demostración}
%\newtheorem{ejer}[teorema]{Ejercicio}
%\newtheorem{problem}[teorema]{Problema}
%\newtheorem{concl}[teorema]{Conclusión}
%
%\providecommand{\abs}[1]{\lvert#1\rvert}
%\providecommand{\sen}[1]{sen #1}
%\providecommand{\norm}[1]{\lVert#1\rVert}
%\providecommand{\ninf}[1]{\norm{#1}_\infty}
%\providecommand{\numn}[1]{\norm{#1}_1}
%\providecommand{\gabs}[1]{\left|{#1}\right|}
%\newcommand{\bor}[1]{\mathcal{B}(#1)}
%\newcommand{\R}{\mathbb{R}}
%\newcommand{\Z}{\mathbb{Z}}
%\newcommand{\N}{\mathbb{N}}
%\newcommand{\Q}{\mathbb{Q}}
%\newcommand{\C}{\mathbb{C}}
%\newcommand{\Pro}{\mathbb{P}}
%\newcommand{\Tau}{\mathcal{T}}
%\newcommand{\verteq}{\rotatebox{90}{$\,=$}}
%\newcommand{\vertequiv}{\rotatebox{110}{$\,\equiv$}}
%\providecommand{\lrg}{\longrightarrow}
%\providecommand{\func}[2]{\colon{#1}\longrightarrow{#2}}
%\newcommand*{\QED}{\hfill\ensuremath{\blacksquare}}
%\newcommand*\circled[1]{\tikz[baseline=(char.base)]{
%            \node[shape=circle,draw,inner sep=1.5pt] (char) {#1};}}
%\newcommand*{\longhookarrow}{\ensuremath{\lhook\joinrel\relbar\joinrel\rightarrow}}


\begin{document}
%\title{Topología de Superficies}
%\author{Antonio Rafael Quintero Toscano\\ Javier Aguilar Martín}
%\date{Curso 2016/2017}
%\maketitle

\author{Javier Aguilar Martín }
\date{\today}
\title{La conjetura de Smale:\\ $Diff(S^3)\simeq O(4)$}

\maketitle


\begin{abstract}
Un guion que no pretende ser totalmente detallado ni riguroso sobre la prueba de Hatcher de la conjetura de Smale. RECUERDA PONER LAS DOS COSAS COMO BIBLIOGRAFÍA
\end{abstract}


	\vfill
	Esta obra está licenciada bajo la Licencia Creative Commons Atribución 3.0 España. Para ver una copia de esta licencia, visite \url{http://creativecommons.org/licenses/by/3.0/es/} o envíe una carta a Creative Commons, PO Box 1866, Mountain View, CA 94042, USA.


\newpage
\tableofcontents

\newpage

\section{Introducción y enunciado equivalente}
\subsection{Introducción}


\subsection{Enunciados equivalentes}\label{equivalentes}
Es conocido que $Diff(S^n)\simeq O(n+1)\times Diff(D^n\ rel\ \partial D^n)$ para todo $n$. Por tanto, la conjetura de Smale $Diff(S^4)\simeq O(4)$ es equivalente a $Diff(D^3\ rel\ \partial D^3)\simeq *$. El teorema puede ser reenunciado como como que para los espacios de smooth embeddings $Emb(D^3,\R^3)$ y $Emb(S^2,\R^n)$, la restricción $\rho:Emb(D^3,\R^3)\to Emb(S^2,\R^n)$ induce sobreyección en $\pi_k$ para todo $k$, pues $\rho$ es una fibración cuyas fibra es precisamente $Diff(D^3\ rel\ \partial D^3)$ (se usa la sucesión exacta larga de homotopía de una fibración). Consideremos ahora el diagrama
\[
\begin{tikzcd}
Emb(D^3,\R^3)\arrow[rr, "\rho"] \arrow[dr, "\simeq"]& & Emb(S^2,\R^n)\arrow[dl]\\
&GL(n,\R)&
\end{tikzcd}
\]
donde la equivalencia de homotopía consiste en evaluar la derivada en un punto. A partir de esto podemos enunciar equivalentemente el teorema como ``el espacio de smoothly embedded 2-esferas en $\R^3$ es contráctil''. Este espacio es el espacio de órbitas $Emb(S^2,\R^3)/Diff(S^2)$. Como $Diff(S^2)\simeq O(3)$, la contractibilidad del espacio es equivalente a que $\rho$ sea equivalencia de homotopía, pues $GL(n,\R)\simeq O(n)$ mediante Gram-Schmidt. Esta última versión será la que se probará.


\section{Normalización, partición de la unidad y cirugía}
\subsection{Normalización}
\begin{defi}
Una superficie orientada $S$ de $\R^3$ se dice \emph{elemental} si verifica las siguientes condiciones:
\begin{enumerate}
\item Cada curva del borde está en un plano horizontal, siendo $S$ vertical en un entorno de cada curva. 
\item La intersección de $S$ con una línea vertical es conexa (posiblemente vacía).
\item Si denotamos por $S^+$ (respectivamente $S^-$) a la adherencia del conjunto de puntos donde la normal tiene coordenaza $z$ estrictamente positiva (respectivamente negativa), $S^+$ y $S^-$ están aisladas la una de la otra. 
\end{enumerate}
\end{defi}

\begin{prop}[de normalización]
Dada una familia de esferas $\Sigma_t$, $t\in S^k$, existe un recubrimiento de $S^k$ por bolas abiertas $B_i$, existen planos horizontales $P_{ij}$ (todos distintos) y una deformación de $\Sigma_t$ a una familia $\Sigma'_t$ tal que:
\begin{enumerate}
\item Para todo $t\in B_i$ y para todo $j$, la parte de $\Sigma'_t$ comprendida entre $P_{ij}$ y $P_{ij+1}$ es una superficie elemental
\item $\Sigma'_t$ es la unión de las partes anteriores. 
\end{enumerate}
\end{prop}
\begin{dem}
Sea $U_i^{\pm}\subseteq\Sigma_t$ el conjunto de puntos donde la normal exterior forma un ángulo $\leq\frac{\pi}{4}$ con el vector $\pm(0,0,1)$. Sea $\rho_t:\Sigma_t\to [0,1]$ una familia de funciones con soporte en $U_i^+\cup U_i^-$ con $\rho_t>0$ en los puntos que tienen plano tangente horizontal. Definimos el campo de vectores $v_t$ sobre $\Sigma_t$ como $grad(h_t)+\rho_t\cdot(0,0,1)$, donde $h_t$ es la función altura de $\Sigma_t$. Por definición, $v_t$ tiene coordenada $z$ positiva en $\Sigma_t$ y puede ser extendido a un campo de vectores en $\R^3$ con la misma propiedad. Sea $f_t$ una familia de difeomorfismos que preserven la altura en $\R^3$ llevando las trayectorias de $v_t$ a líneas verticales de $\R^3$. 

Existe $\delta\geq 0$, independiente de $t$, tal que todo segmento vertical en $\R^3$ de longitud menor o igual que $\delta$ interseca con $\Sigma'_t=f_t(\Sigma_t)$ en un conjunto conexo. Esto se sigue porque el campo de vectores $(0,0,1)$ sobre $\Sigma'_t$ solo puede apuntar a un lado de $\Sigma'_t$ localmente. Esto es, $(0,0,1)$ apunta hacia fuera en $f_t(U_i^+)$ y hacia dentro en $f_t(U_i^-)$, estando estos conjuntos separados por una distancia $d>0$ independiente de $t$. Podemos suponer que $\delta<\frac{1}{2}d$. 

Para cada $t$ podemos elegir un número finito de planos horizontales $P_{ij}$ transversales a $\Sigma'_t$ de modo que planos adyacentes estén a distancia estrictamente menor que $\delta$ y que $\Sigma'_t$ se encuentre entre los $P_{ij}$ extremos. Estos planos se mantienen transversales a $\Sigma'_t$ en alguna bola abierta $B_i\subseteq S^k$. Por compacidad de $S^k$, podemos recubir $S^k$ con un número finito de estas bolas. Podemos suponer que los $P_{ij}$ son distintos sin problema pues se escogen de un conjunto abierto de planos horizontales. 

Por construcción, para cada $i$ y $t\in B_i$, la parte de $\Sigma'_t$ entre planos adyacentes son superficies elementales ya que hemos tomado $\delta<\frac{1}{2}d$ y $\Sigma'_t$ es la unión de estas superficies
\QED
\end{dem}

A partir de ahora nos referiremos a $\Sigma'_t$ como $\Sigma_t$. 

\subsection{Función de Hatcher}
Sea $C_t$ la unión de las curvas $\Sigma_t\cap P_{ij}$ y $C=\bigcup_t C_t$. La unión $C$ tiene una topología natural: podemos seguir continuamente la deformación de $c_t$, componente de $\Sigma_t\cap P_{ij}$, a medida que $t$ recorre $B_i$. Podemos definir un orden parcial $c_t>c'_t$ si $c_t$ y $c'_t$ están en el mismo $P_{ij}$ y $c_t$ rodea a $c'_t$. 

\begin{prop}
Existe una función (de Hatcher) $\lambda:\pi_0(C)\to (0,1)$ inyectiva y monónota decreciente, es decir, $c_t>c'_t$ implica $\lambda(c_t)<\lambda(c'_t)$.
\end{prop}
\begin{dem}
CON LOS DIBUJOS DE LOS CÍRCULOS ALGUNOS METIDOS EN OTROS HACER EL GRAFO Y DEFINIR PARA UN T FIJO LA FUNCION COGIENDO INTERVALOS PARA CADA NIVEL, YA QUE HAY UNA CANTIDAD FINITA. COMO TAMBIÉN HAY UNA CANTIDAD FINITA DE BOLAS Y PARA LOS DE UNA MISMA BOLA LOS GRAFOS SON IGUALES Y SOLO CAMBIAN AL CAMBIAR LA BOLA PORQUE LOS PLANOS SON DISTINTOS ENTONCES SIGUE HABIENDO UNA CANTIDAD FINITA Y SE PUEDE APAÑAR
\QED
\end{dem}
\subsection{Partición de la unidad}
Elegimos $\varepsilon>0$ tal que para $c_t\neq c'_t$ se tiene $|\lambda(c_t)-\lambda(c'_t)|>2\varepsilon$, y $\delta>0$ tal que dos planos $P_{ij}$ distan de al menos $2\delta$ y que $\Sigma_t$ sea vertical en un $\delta$-entorno de $P_{ij}$. Sean $\Gamma_i=\{(t,\lambda(c_t))\in B_i\times (0,1)\}$ y $\widetilde{\Gamma}_i$ un $\varepsilon$-entorno de $\Gamma_i$. Para cada $i$, se define $T_i\cong S^{k-1}\times[0,1]\subset S^k\times[0,1]$ como como un anillo que une $\partial B_i\times\{1\}$ con $\partial B'_i\times\{0\}$, siendo $B'_i$ bolas abiertas contenidas en las $B_i$ y que recubren $S^k$. Cada $\widetilde{\Gamma}_i\cap T_i$ es una unión de subanillos cuyas proyecciones sobre $S^k$ denotamos $Z_{il}$. Denotamos $E_i$ a $\widetilde{\Gamma}_i$ truncado por $B'_i$, unión de bandas tal como muestra la siguiente figura

HACER EL DIBUJO ROTADO EN LA IZQUIERDA Bi CON B'i DENTRO, RECORDAR QUE HE DEFINIDO LAMBDA DE MODO QUE ES CONSTANTE Y POR ESO SALEN LÍNEAS VERTICALES. QUIZÁ INCLUIR til EN EL DIBUJO

Podemos decir que $E=\bigcup E_i$ juega el papel de una partición de la unidad, pues $\partial E$ induce una partición de $t\times [0,1]$ variando continuamente con $t$. Sea $X_i$ la proyección sobre $S^k$ de $\Gamma_i\cap T_i$; $X_i=\bigcup X_{il}$, donde $X_{il}$ es el alma de $Z_{il}\cong X_{il}\times [-1,1]$. La coordenada transversal $t_{il}$ crece en el sentido que va de $B'_i$ a $B_i$. 

Nótese que, para $i$ fijo, los $Z_{il}$ son disjuntos dos a dos por el $\varepsilon$ que se ha elegido. Se tomará un recubrimiento $\{B_i\}$ tal que cada $t$ pertenezca a lo sumo a $k+1$ bolas. Se puedeN elegir los $B'_i$ de forma que la unión de los $t$ que pertenece exactamente a $q$ $Z_{il}$, $1\leq q\leq k$, sea una unión disjunta de \emph{cubos} que tengan por alma una componente de la intersección de $q$ $X_{il}$, con los $t_{il}$ respectivas como coordenadas transversales, y que ningún $t$ pertenezca a $k+1$ $Z_{il}$. 

\subsection{Cirugía}
Para cada $t$ fijo, construimos los objetos $\Sigma_{tu}$. Los valores $\lambda(c_t)$ se ordenan en $u^1=\lambda(c_t^1)>u^2=\lambda(c_t^2)>\cdots$. Sea $\Delta(c_t)$ el disco horizontal bordeado por $c_t$. Para $u>u^1$, $\Sigma_{tu}=\Sigma_t$. Para $u=u^1$, $\Sigma_{tu}=\Sigma_t\cup \Delta(c_t^1)$ (aparición de un disco de cirugía). Si $u\in[u^1, u^1-\varepsilon]$, ejecutamos la cirugía: $\Delta(c_t^1)$ es reemplazado por dos discos $\Delta^+$ y $\Delta^-$ que se separan progresivamente, uno hacia arriba y otro hacia abajo hasta una distancia de $\Delta(c_t^1)$ igual a $\lambda(c_t)\cdot\delta$; al mismo tiempo retiramos el anillo vertical entre $\partial\Delta^+$ y $\partial\Delta^-$. Después de esto no añadimos ninguna cirugía hasta $u=u^2$ y repetimos. 

PONER POR AQUÍ EL DIBUJO DE HATCHER Y ALOMEJOR TAMBIÉN EL DE BOURBAKI

Hay una manera de describir $\Sigma_{tu}$ de modo único como unión de esferas que tienen solamente un disco horizontal en común dos a dos. Cada una de ellas es un \emph{factor} y sus discos horizontales se llaman \emph{caras}. Una esfera que es unión de superficies elementales con discos horizontales se dice \emph{primitiva}. Cada componente de $\Sigma_{t0}$ es una esfera primitiva a la que se le han añadido discos de cirugía. Si $\Delta$ es una cara común a dos factores $\Sigma_1$ y $\Sigma_2$, se dice que $\Delta$ es una cara \emph{suma} si $\overline{\Sigma}_1$ y $\overline{\Sigma}_2$ son bolas que intersecan solo en $\Delta$, o \emph{diferencia} si hay una relación de inclusión entre estas bolas. 

PONER EL DIBUJO 

\section{Contornos}

 Si $\Sigma\subset\R^3$ es una esfera encerrando una bola cerrada $\overline{\Sigma}$. Definimos su \emph{contorno} como $C(\Sigma)$, el espacio cociente $\overline{\Sigma}/\sim$ de la relación $x\sim y$ si hay un segmento vertical contenido en $\overline{\Sigma}$ uniendo $x$ e $y$. En otras palabras, $C(\Sigma)$ es el espacio de hojas de la foliación de $\overline{\Sigma}$ mediante líneas verticales. La aplicación cociente se denotará $C:\overline{\Sigma}\to C(\Sigma)$. Nótese que $C$ restringida a $\Sigma$ es sobreyectiva. 
 
 EJEMPLO 5.1 DE HATCHER
 
 Motivados por esta construcción damos las siguientes definiciones.
 
 \begin{defi}
 Sean $n$ funciones de clase $C^1$, $f_1,\dots, f_n:[0,1]\to\R$ nulas en el borde y verificando que si $f_i(x)=f_j(x)$ para algún $x\in [0,1]$ entonces $f_i'(x)=f_j'(x)$. La figura formada por sus grafos se llamará \emph{bloque de bigones}. Un \emph{sistema de bigones} es una unión finita de bloques disjuntos. Para $n=2$, la adherencia de la unión de las componentes acotadas se llama una \emph{lengua}. 
 \end{defi}

\begin{defi}\label{lenguas}
Una \emph{estructura de lenguas} $\mathcal{L}$ es una superficie con borde $L$, dotada de una proyección $\pi:L\to\R^2$, un disco inicial $D_0$ y una colección finita de lenguas $T_1,\dots, T_n$ sobre $L$ con las condiciones siguientes:
\begin{enumerate}
\item $L$ es unión de discos lisos $D_1,\dots, D_n$ con $D_0$, tales que sobre cada interior de los cuales $\pi$ es una sumersión. 
\item $\pi(D_i)\cap\pi\left(\bigcup_{j<i}D_j\right)$ es un subdisco $d_i$ de $\pi(D_i)$ tal que $d_i\cap\pi(D_i)$ contiene al menos un arco. 
\item $T_i=adh(\pi(D_i)-d_i)$, $int(T_i)\cap int(T_j)=\emptyset$ y $D_0\cap int(T_i)=\emptyset$. El lado de unión de $T_i$ es $T_i\cap d_i$ y su lado libre es $adh(\partial T_i-\partial d_i)$. 
\item Todo punto del lado de unión pertenece al lado libre de alguna otra lengua o a $\partial D_0$.  
\end{enumerate}
\end{defi}

Una lengua no tiene por qué ser conexa, por ello podemos hablar de \emph{subdivisión} de una estructura de lenguas, en la que se eligen como lenguas distintas algunas de las componentes de una lengua existente. 

\begin{teorema}
Si $\Sigma$ es una esfera primitiva, el contorno $C(\Sigma)$ admite una estructura de lenguas. PONER LA REFERENCIA (ES LA PROPOSICIÓN 5.1)
\end{teorema}

DIBUJO 3.1 DE BOURBAKI

\begin{nota}\
\begin{enumerate}
\item La definición \ref{lenguas} se extiende a familias de esferas.
\item El orden de unión de las lenguas no forma parte de la estructura, pero para una familia se puede dar un orden localmente en el espacio de parámetros.  
\end{enumerate}
\end{nota}

Sea $\Sigma$ una esfera primitiva y $\Delta_1,\dots, \Delta_n$ una colección de discos de cirugía en planos horizontales distintos, con $int(\Delta_i)\cap\Sigma=\emptyset$. Tenemos $\Sigma\cup\Delta_1\cup\cdots\cup\Delta_n$ como unión de factores $\Sigma_0\cup\cdots\cup \Sigma_n$ con $\Delta_k$ cara de $\Sigma_k$ para $k\geq 1$. Suponemos que $\Sigma_0$ es \emph{grande}, esto es, que $\pi(\Sigma_0)\cap\partial\Sigma$ contiene un arco. Para una familia $\Sigma_t$ pedimos que $\pi(\Sigma_{0t})\cap\partial\Sigma_t$ contenga localmente arcos que varíen suavemente con $t$.

\begin{teorema}[de compatibilidad]PONER REFERENCIA (ESTÁ EN BOURBAKI) 
Existe una estructura de lenguas $\mathcal{L}$ (resp. $\mathcal{L}_i$) sobre $C(\Sigma)$ (resp. $\Sigma_i$) verificando las condiciones siguientes:
\begin{enumerate}
\item Para $k\geq 1$, $C(\Delta_k)$ es el disco inicial de $\mathcal{L}_k$. El disco inicial $D_0$ de $\mathcal{L}_0$ es también disco inicial de $\mathcal{L}$.
\item La unión de los $\pi(\partial T_i)$ es un sistema de bigones.
\item Para $j\neq j'$, $int(T_j)\cap int(T_{j'})=\emptyset$ o bien hay una relación de contención. 
\end{enumerate}
\end{teorema}

En presencia de un parámetro el enunciado es el mismo salvo que el factor inicial $\Sigma_{0t}$ no puede ser seguido continuamente en general. Por ejemplo si tomamos como factor inicial uno tal que la proyección tenga la forma de la FIGURA DE LA LENGUA QUE SOBRE SALE y hacemos que varíe hasta que deje de sobresalir. Para arreglarlo cambiamos de factor a lo largo de una subvariedad $V$ de codimensión 1 del espacio de parámetros; a un lado el punto crítico elegimos $\Sigma_{0t}=\Sigma'_t$ y al otro lado elegimos $\Sigma_{0t}=\Sigma''_t$. Se verifica que $\Sigma'_t$ y $\Sigma''_t$ tienen una cara común de tipo suma $\Delta_t$. Si para $t\in V$ tomamos $C(\Delta_t)$ como disco inicial de $\mathcal{L}$, entonces el teorema de compatibilidad se preserva en presencia de un parámetro. 

Antes de seguir avanzando enunciamos un hecho esencial: una familia de esferas primitivas es homótopa a cero dentro del espacio de todas las esferas. Esto será establecido en la siguiente sección gracias a la estructura de contorno. En cierta forma, el problema se vuelve 2-dimensional y admite solución gracias al ``pequeño teorema de Smale'' REFERENCIA: $Diff(D^2\ rel\ \partial D^2)$ es contráctil. Finalmente, el teorema en dimensión 3 aparece como consecuencia del teorema en dimensión 2.

\section{Aplastamientos}

\subsection{Compatibilidad con la suma}
\subsection{Compatibilidad con la diferencia}

\section{Cómo hacer continua la familia $\Sigma_{to}$}
\subsection{Orden en las caras}
\subsection{Conclusión}


\section{Prueba de la conjetura de Smale}



\end{document}