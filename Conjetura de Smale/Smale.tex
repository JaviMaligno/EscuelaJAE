\documentclass[twoside, 11pt]{article}
\usepackage{../estilo-apuntes}
%\usepackage{amsmath,amssymb}
%\usepackage[utf8]{inputenc}
%\usepackage[spanish]{babel}
%\usepackage{caption}
%\usepackage[]{graphicx}
%\usepackage{enumerate}
%\usepackage{amsthm}
%\usepackage{tikz-cd}
%\usetikzlibrary{babel}
%\usepackage{pgf,tikz}
%\usepackage{mathrsfs}
%\usepackage{bm}  
%\usetikzlibrary{arrows}
%\usetikzlibrary{cd}
%\usepackage[spanish]{babel}
%\usepackage{fancyhdr}
%\usepackage{titlesec}
%\usepackage{floatrow}
%\usepackage{makeidx}
%\usepackage[tocflat]{tocstyle}
%\usetocstyle{standard}
%\usepackage{subfiles}
%\usepackage{color}  
%\usepackage{hyperref}
%\hypersetup{colorlinks=true,citecolor=red, linkcolor=blue}
%\usepackage{eurosym}
%%\usepackage{ntheorem}
%
%
%\renewcommand{\baselinestretch}{1,4}
%\setlength{\oddsidemargin}{0.25in}
%\setlength{\evensidemargin}{0.25in}
%\setlength{\textwidth}{6in}
%\setlength{\topmargin}{0.1in}
%\setlength{\headheight}{0.1in}
%\setlength{\headsep}{0.1in}
%\setlength{\textheight}{8in}
%\setlength{\footskip}{0.75in}
%
%\theoremstyle{definition}
%
%\newtheorem{teorema}{Teorema}[section]
%\newtheorem{defi}[teorema]{Definición}
%\newtheorem{coro}[teorema]{Corolario}
%\newtheorem{lemma}[teorema]{Lema}
%\newtheorem{ej}[teorema]{Ejemplo}
%\newtheorem{ejs}[teorema]{Ejemplos}
%\newtheorem{observacion}[teorema]{Observación}
%\newtheorem{observaciones}[teorema]{Observaciones}
%\newtheorem{prop}[teorema]{Proposición}
%\newtheorem{propi}[teorema]{Propiedades}
%\newtheorem{nota}[teorema]{Nota}
%\newtheorem{notas}[teorema]{Notas}
%\newtheorem*{dem}{Demostración}
%\newtheorem{ejer}[teorema]{Ejercicio}
%\newtheorem{problem}[teorema]{Problema}
%\newtheorem{concl}[teorema]{Conclusión}
%
%\providecommand{\abs}[1]{\lvert#1\rvert}
%\providecommand{\sen}[1]{sen #1}
%\providecommand{\norm}[1]{\lVert#1\rVert}
%\providecommand{\ninf}[1]{\norm{#1}_\infty}
%\providecommand{\numn}[1]{\norm{#1}_1}
%\providecommand{\gabs}[1]{\left|{#1}\right|}
%\newcommand{\bor}[1]{\mathcal{B}(#1)}
%\newcommand{\R}{\mathbb{R}}
%\newcommand{\Z}{\mathbb{Z}}
%\newcommand{\N}{\mathbb{N}}
%\newcommand{\Q}{\mathbb{Q}}
%\newcommand{\C}{\mathbb{C}}
%\newcommand{\Pro}{\mathbb{P}}
%\newcommand{\Tau}{\mathcal{T}}
%\newcommand{\verteq}{\rotatebox{90}{$\,=$}}
%\newcommand{\vertequiv}{\rotatebox{110}{$\,\equiv$}}
%\providecommand{\lrg}{\longrightarrow}
%\providecommand{\func}[2]{\colon{#1}\longrightarrow{#2}}
%\newcommand*{\QED}{\hfill\ensuremath{\blacksquare}}
%\newcommand*\circled[1]{\tikz[baseline=(char.base)]{
%            \node[shape=circle,draw,inner sep=1.5pt] (char) {#1};}}
%\newcommand*{\longhookarrow}{\ensuremath{\lhook\joinrel\relbar\joinrel\rightarrow}}


\begin{document}
%\title{Topología de Superficies}
%\author{Antonio Rafael Quintero Toscano\\ Javier Aguilar Martín}
%\date{Curso 2016/2017}
%\maketitle

\author{Javier Aguilar Martín }
\date{\today}
\title{La conjetura de Smale:\\ $Diff(S^3)\simeq O(4)$}

\maketitle


\begin{abstract}
Un guion que no pretende ser totalmente detallado ni riguroso sobre la prueba de Hatcher de la conjetura de Smale. RECUERDA PONER LAS DOS COSAS COMO BIBLIOGRAFÍA
\end{abstract}


	\vfill
	Esta obra está licenciada bajo la Licencia Creative Commons Atribución 3.0 España. Para ver una copia de esta licencia, visite \url{http://creativecommons.org/licenses/by/3.0/es/} o envíe una carta a Creative Commons, PO Box 1866, Mountain View, CA 94042, USA.


\newpage
\tableofcontents

\newpage

\section{Introducción y enunciado equivalente}
\subsection{Introducción}


\subsection{Enunciados equivalentes}
Es conocido que $Diff(S^n)\simeq O(n+1)\times Diff(D^n\ rel\ \partial D^n)$ para todo $n$. Por tanto, la conjetura de Smale $Diff(S^4)\simeq O(4)$ es equivalente a $Diff(D^3\ rel\ \partial D^3)\simeq *$. El teorema puede ser reenunciado como como que para los espacios de smooth embeddings $Emb(D^3,\R^3)$ y $Emb(S^2,\R^n)$, la restricción $\rho:Emb(D^3,\R^3)\to Emb(S^2,\R^n)$ induce sobreyección en $\pi_k$ para todo $k$, pues $\rho$ es una fibración cuyas fibra es precisamente $Diff(D^3\ rel\ \partial D^3)$ (se usa la sucesión exacta larga de homotopía de una fibración). Consideremos ahora el diagrama
\[
\begin{tikzcd}
Emb(D^3,\R^3)\arrow[rr, "\rho"] \arrow[dr, "\simeq"]& & Emb(S^2,\R^n)\arrow[dl]\\
&GL(n,\R)&
\end{tikzcd}
\]
donde la equivalencia de homotopía consiste en evaluar la derivada en un punto. A partir de esto podemos enunciar equivalentemente el teorema como ``el espacio de smoothly embedded 2-esferas en $\R^3$ es contráctil''. Este espacio es el espacio de órbitas $Emb(S^2,\R^3)/Diff(S^2)$. Como $Diff(S^2)\simeq O(3)$, la contractibilidad del espacio es equivalente a que $\rho$ sea equivalencia de homotopía, pues $GL(n,\R)\simeq O(n)$ mediante Gram-Schmidt. Esta última versión será la que se probará.


\section{Normalización, partición de la unidad y cirugía}
\subsection{Normalización}
\begin{defi}
Una superficie orientada $S$ de $\R^3$ se dice \emph{elemental} si verifica las siguientes condiciones:
\begin{enumerate}
\item Cada curva del borde está en un plano horizontal, siendo $S$ vertical en un entorno de cada curva. 
\item La intersección de $S$ con una línea vertical es conexa (posiblemente vacía).
\item Si denotamos por $S^+$ (respectivamente $S^-$) a la adherencia del conjunto de puntos donde la normal tiene coordenaza $z$ estrictamente positiva (respectivamente negativa), $S^+$ y $S^-$ están aisladas la una de la otra. 
\end{enumerate}
\end{defi}

\begin{prop}[de normalización]
Dada una familia de esferas $\Sigma_t$, $t\in S^k$, existe un recubrimiento de $S^k$ por bolas abiertas $B_i$, existen planos horizontales $P_{ij}$ (todos distintos) y una deformación de $\Sigma_t$ a una familia $\Sigma'_t$ tal que:
\begin{enumerate}
\item Para todo $t\in B_i$ y para todo $j$, la parte de $\Sigma'_t$ comprendida entre $P_{ij}$ y $P_{ij+1}$ es una superficie elemental
\item $\Sigma'_t$ es la unión de las partes anteriores. 
\end{enumerate}
\end{prop}
\begin{dem}
Sea $U_i^{\pm}\subseteq\Sigma_t$ el conjunto de puntos donde la normal exterior forma un ángulo $\leq\frac{\pi}{4}$ con el vector $\pm(0,0,1)$. Sea $\rho_t:\Sigma_t\to [0,1]$ una familia de funciones con soporte en $U_i^+\cup U_i^-$ con $\rho_t>0$ en los puntos que tienen plano tangente horizontal. Definimos el campo de vectores $v_t$ sobre $\Sigma_t$ como $grad(h_t)+\rho_t\cdot(0,0,1)$, donde $h_t$ es la función altura de $\Sigma_t$. Por definición, $v_t$ tiene coordenada $z$ positiva en $\Sigma_t$ y puede ser extendido a un campo de vectores en $\R^3$ con la misma propiedad. Sea $f_t$ una familia de difeomorfismos que preserven la altura en $\R^3$ llevando las trayectorias de $v_t$ a líneas verticales de $\R^3$. 

Existe $\delta\geq 0$, independiente de $t$, tal que todo segmento vertical en $\R^3$ de longitud menor o igual que $\delta$ interseca con $\Sigma'_t=f_t(\Sigma_t)$ en un conjunto conexo. Esto se sigue porque el campo de vectores $(0,0,1)$ sobre $\Sigma'_t$ solo puede apuntar a un lado de $\Sigma'_t$ localmente. Esto es, $(0,0,1)$ apunta hacia fuera en $f_t(U_i^+)$ y hacia dentro en $f_t(U_i^-)$, estando estos conjuntos separados por una distancia $d>0$ independiente de $t$. Podemos suponer que $\delta<\frac{1}{2}d$. 

Para cada $t$ podemos elegir un número finito de planos horizontales $P_{ij}$ transversales a $\Sigma'_t$ de modo que planos adyacentes estén a distancia estrictamente menor que $\delta$ y que $\Sigma'_t$ se encuentre entre los $P_{ij}$ extremos. Estos planos se mantienen transversales a $\Sigma'_t$ en alguna bola abierta $B_i\subseteq S^k$. Por compacidad de $S^k$, podemos recubir $S^k$ con un número finito de estas bolas. Podemos suponer que los $P_{ij}$ son distintos sin problema pues se escogen de un conjunto abierto de planos horizontales. 

Por construcción, para cada $i$ y $t\in B_i$, la parte de $\Sigma'_t$ entre planos adyacentes son superficies elementales ya que hemos tomado $\delta<\frac{1}{2}d$ y $\Sigma'_t$ es la unión de estas superficies
\QED
\end{dem}

A partir de ahora nos referiremos a $\Sigma'_t$ como $\Sigma_t$. 

\subsection{Función de Hatcher}
Sea $C_t$ la unión de las curvas $\Sigma_t\cap P_{ij}$ y $C=\cup_t C_t$. La unión $C$ tiene una topología natural: podemos seguir continuamente la deformación de $c_t$, componente de $\Sigma_t\cap P_{ij}$, a medida que $t$ recorre $B_i$. Podemos definir un orden parcial $c_t>c'_t$ si $c_t$ y $c'_t$ están en el mismo $P_{ij}$ y $c_t$ rodea a $c'_t$. 

\begin{prop}
Existe una función (de Hatcher) $\lambda:\pi_0(C)\to (0,1)$ inyectiva y monónota decreciente, es decir, $c_t>c'_t$ implica $\lambda(c_t)<\lambda(c'_t)$.
\end{prop}
\begin{dem}
CON LOS DIBUJOS DE LOS CÍRCULOS ALGUNOS METIDOS EN OTROS HACER EL GRAFO Y DEFINIR PARA UN T FIJO LA FUNCION COGIENDO INTERVALOS PARA CADA NIVEL, YA QUE HAY UNA CANTIDAD FINITA. COMO TAMBIÉN HAY UNA CANTIDAD FINITA DE BOLAS Y PARA LOS DE UNA MISMA BOLA LOS GRAFOS SON IGUALES Y SOLO CAMBIAN AL CAMBIAR LA BOLA PORQUE LOS PLANOS SON DISTINTOS ENTONCES SIGUE HABIENDO UNA CANTIDAD FINITA Y SE PUEDE APAÑAR
\QED
\end{dem}
\subsection{Partición de la unidad}
Elegimos $\varepsilon>0$ tal que para $c_t\neq c'_t$ se tiene $|\lambda(c_t)-\lambda(c'_t)|>2\varepsilon$, y $\delta>0$ tal que dos planos $P_{ij}$ distan de al menos $2\delta$ y que $\Sigma_t$ sea vertical en un $\delta$-entorno de $P_{ij}$. Sean $\Gamma_i=\{(t,\lambda(c_t))\in B_i\times (0,1)\}$ y $\widetilde{\Gamma}_i$ un $\varepsilon$-entorno de $\Gamma_i$. Para cada $i$, se define $T_i\cong S^{k-1}\times[0,1]\subset S^k\times[0,1]$ como como un anillo que une $\partial B_i\times\{1\}$ con $\partial B'_i\times\{0\}$, siendo $B'_i$ bolas abiertas contenidas en las $B_i$ y que recubren $S^k$. Cada $\widetilde{\Gamma}_i\cap T_i$ es una unión de subanillos cuyas proyecciones sobre $S^k$ denotamos $Z_{il}$. Denotamos $E_i$ a $\widetilde{\Gamma}_i$ truncado por $B'_i$, unión de bandas tal como muestra la siguiente figura

HACER EL DIBUJO ROTADO EN LA IZQUIERDA Bi CON B'i DENTRO, RECORDAR QUE HE DEFINIDO LAMBDA DE MODO QUE ES CONSTANTE Y POR ESO SALEN LÍNEAS VERTICALES. QUIZÁ INCLUIR til EN EL DIBUJO

Podemos decir que $E=\bigcup E_i$ juega el papel de una partición de la unidad, pues $\partial E$ induce una partición de $t\times [0,1]$ variando continuamente con $t$. Sea $X_i$ la proyección sobre $S^k$ de $\Gamma_i\cap T_i$; $X_i=\bigcup X_{il}$, donde $X_{il}$ es el alma de $Z_{il}\cong X_{il}\times [-1,1]$. La coordenada transversal $t_{il}$ crece en el sentido que va de $B'_i$ a $B_i$. 

Nótese que, para $i$ fijo, los $Z_{il}$ son disjuntos dos a dos por el $\varepsilon$ que se ha elegido. Se tomará un recubrimiento $\{B_i\}$ tal que cada $t$ pertenezca a lo sumo a $k+1$ bolas. Se puedeN elegir los $B'_i$ de forma que la unión de los $t$ que pertenece exactamente a $q$ $Z_{il}$, $1\leq q\leq k$, sea una unión disjunta de \emph{cubos} que tengan por alma una componente de la intersección de $q$ $X_{il}$, con los $t_{il}$ respectivas como coordenadas transversales, y que ningún $t$ pertenezca a $k+1$ $Z_{il}$. 
\subsection{Cirugía}

\section{Contornos}
\subsection{Compatibilidad con la suma}
\subsection{Compatibilidad con la diferencia}

\section{Cómo hace continua la familia $\Sigma_{to}$}
\subsection{Orden en las caras}
\subsection{Conclusión}


\section{Prueba de la conjetura de Smale}



\end{document}